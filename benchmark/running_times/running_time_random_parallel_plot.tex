\documentclass{article}

\usepackage[utf8]{inputenc}
\usepackage[english]{babel}

\usepackage{amsmath,amsfonts,amssymb}
\usepackage{fullpage}
\usepackage{verbatim}

\usepackage{xcolor}
\usepackage{morefloats}

\usepackage{pgfplots}
\usepgfplotslibrary{
  units,
  groupplots
}
\pgfplotsset{compat=1.5}
\makeatletter
\newcommand\gpsubtitle[1]{(\@alph{\pgfplots@group@current@plot}) #1}
\makeatother

\usepackage{placeins}

\usepackage{pgfplotstable}
\usepackage{subfig}
\usepackage{xspace}

\newcommand{\compssssort}{\textsf{S$^4$o}\xspace}
\newcommand{\compmyssssaxtmann}{\textsf{1S$^4$o}\xspace}
\newcommand{\compssssschneider}{\textsf{S$^4$oS}\xspace}
\newcommand{\compissssort}{\textsf{I1S$^4$o}\xspace}
\newcommand{\compblock}{\textsf{BlockQ}\xspace}
\newcommand{\compspdq}{\textsf{BlockPDQ}\xspace}
\newcommand{\compssort}{\textsf{std::sort}\xspace}
\newcommand{\compjdksyaros}{\textsf{UnusedDualPivot}\xspace}
\newcommand{\compsyaros}{\textsf{DualPivot}\xspace}
\newcommand{\compswiki}{\textsf{WikiSort}\xspace}
\newcommand{\compsgrail}{\textsf{GrailSort}\xspace}
\newcommand{\compsmergequick}{\textsf{QMSort}\xspace}
\newcommand{\compstim}{\textsf{Timsort}\xspace}
\newcommand{\radixlearned}{\textsf{LearnedSort}\xspace}

\newcommand{\radixsska}{\textsf{SkaSort}\xspace}
\newcommand{\compissrsort}{\textsf{I1S$^2$ro}\xspace}

\newcommand{\inplacemsdradixsort}{\textsf{IMSDradix}\xspace}
\newcommand{\compiparassssort}{\textsf{IPS$^4$o}\xspace}
\newcommand{\compiparassssortnts}{\textsf{IPS$^4$oNT}\xspace}
\newcommand{\compparastringssssort}{\textsf{StringPS$^4$o}\xspace}
\newcommand{\compmyparassssaxtmann}{\textsf{PS$^4$o}\xspace}
\newcommand{\comppsort}{\textsf{MCSTLmwm}\xspace}
\newcommand{\comppbalancedsort}{\textsf{MCSTLbq}\xspace}
\newcommand{\comppunbalancedsort}{\textsf{MCSTLubq}\xspace}
\newcommand{\compppbbs}{\textsf{PBBS}\xspace}
\newcommand{\compptbb}{\textsf{TBB}\xspace}
\newcommand{\radixregion}{\textsf{RegionSort}\xspace}
\newcommand{\radixipp}{\textsf{IppRadix}\xspace}
\newcommand{\comppaspas}{\textsf{ASPaS}\xspace}
\newcommand{\radixraduls}{\textsf{RADULS2}\xspace}
\newcommand{\radixparadis}{\textsf{PARADIS}\xspace}

\newcommand{\radixppbbr}{\textsf{PBBR}\xspace}
\newcommand{\compiparassrsort}{\textsf{IPS$^2$ro}\xspace}

\newcommand{\pcintellargefour}{\mbox{I4x20}\xspace}
\newcommand{\pcintelfour}{\mbox{A1x64}\xspace}
\newcommand{\pcinteltwo}{\mbox{I2x16}\xspace}
\newcommand{\pcamd}{\mbox{A1x16}\xspace}

\newcommand{\distzipf}{Zipf\xspace}
\newcommand{\distuniform}{Uniform\xspace}
\newcommand{\distexpo}{Exponential\xspace}
\newcommand{\distalmostsorted}{AlmostSorted\xspace}
\newcommand{\distsorted}{Sorted\xspace}
\newcommand{\distreversesorted}{ReverseSorted\xspace}
\newcommand{\distones}{Ones\xspace}
\newcommand{\distduplicatesroot}{RootDup\xspace}
\newcommand{\distduplicatestwice}{TwoDup\xspace}
\newcommand{\distduplicateseight}{EightDup\xspace}

\newcommand{\bytes}{100B\xspace}
\newcommand{\pair}{Pair\xspace}
\newcommand{\quartet}{Quartet\xspace}
\newcommand{\double}{double\xspace}
\newcommand{\uint}{uint32\xspace}
\newcommand{\ulong}{uint64\xspace}


\newcommand{\todo}[1]{{\color{green} \textbf{TODO:}\textit{#1}}}
%\newcommand{\todo}[1]{}
\newcommand{\answer}[1]{{\color{brown} \textbf{ANSWER:}\textit{#1}}}
%\newcommand{\answer}[1]{}
\newcommand{\frage}[1]{{\color{blue} \textbf{FRAGE:}\textit{#1}}}
%\newcommand{\frage}[1]{}
\newcommand{\note}[1]{{\color{red} \textbf{NOTE:}\textit{#1}} }
%\newcommand{\note}[1]{}
%\newcommand{\changed}[1]{{\color{red}#1}}
\newcommand{\changed}[1]{{#1}}


\newsubfloat{figure}

%%%%%%%%%%%%%%%%%%%%%%%%%%%%%%%%%%%%%%%%%%%%%%%%%%%%%%%%%%%%%%%%%%%%%%%%%%%%%%%%

\title{Parallel Running Times}


\begin{document}

\pgfplotscreateplotcyclelist{my exotic parallel}{%
  black,every mark/.append style={solid,fill=gray},mark=otimes*,mark size=1.75pt\\%
  orange,every mark/.append style={fill=orange!80!black},mark=triangle*,mark size=1.75pt\\%
  yellow!60!black,dashed,every mark/.append style={solid,fill=yellow!80!black},mark=square*,mark size=1.4pt\\%
  teal,every mark/.append style={fill=teal!80!black},mark=star,mark size=1.75pt\\%
  brown!60!black,every mark/.append style={fill=brown!80!black},mark=square*,mark size=1.4pt\\%
  blue,mark=star,every mark/.append style=solid,mark size=1.75pt\\%
  red!70!white,every mark/.append style={solid,fill=red!80!black},mark=*,mark size=1.4pt\\%
  red,dashed,mark=star,mark size=1.75pt\\%
  red,dashed,every mark/.append style={solid,fill=red!80!black},mark=diamond*,mark size=1.75pt\\%
  red!60!black,dashed,every mark/.append style={solid,fill=red!80!black},mark=square*,mark size=1.4pt\\%
}

\pgfplotsset{
  plotstyleparallel/.style={
    cycle list name=my exotic parallel,
  }
}

%% SQL
%% drop view if exists p CASCADE;
%% create view p as
%% select * from pradixalgoswithips4oml
%% union
%% select * from pcomparisonalgos

%% SQL
%% drop view if exists palgos1 CASCADE;
%% create view palgos1 as
%% select avgparallel.* from avgparallel inner join p
%% on avgparallel.algo = p.algo

%% SQL
%% drop view if exists palgos2 CASCADE;
%% create view palgos2 as
%% select * from palgos1 natural join pfast

%% SQL
%% drop view if exists palgos CASCADE;
%% create view palgos as
%% select machine, gen, datatype, algo, parallel, threads, vector, size, meminterleaved, copyback, milli
%% from palgos2
%% where gen like 'random'

%% SQL
%% drop view if exists pavg CASCADE;
%% create view pavg as
%% select machine, threads, algo, size, datatype, AVG(milli) as milli
%% from palgos
%% group by machine, threads, algo, size, datatype

%% SQL
%% drop view if exists pavgnames CASCADE;
%% create view pavgnames as
%% select pavg.*, titles.title, titles.titleorder from pavg
%% inner join titles
%% on titles.algo like pavg.algo

\begin{figure}[htb]
  \begin{tikzpicture}
    \begin{groupplot}[
      group style={
        group size=2 by 2,
        vertical sep=1.2cm,
      },
      width=0.5\textwidth,
      height=0.3\textheight,
      ymax=2.6,
      ytick={0, 0.5, 1, 1.5, 2.0, 2.5},
      xmode=log,
      log base x=2,
      legend columns = 4,
      plotstyleparallel,
      xmajorgrids=true,
      ymajorgrids=true,
      ymin=0,
      xmin=2^16,
      xmax=2^34,
      restrict y to domain=-1:30,
      every axis title/.append style={yshift=-0.2cm},
      clip mode=individual,
      ]
      
      \nextgroupplot[
      title=\pcintellargefour]
    %% MULTIPLOT(algo|ptitle) select title as ptitle, size as x, 1000000.0 * milli * threads / (8 * size * log(2, size)) as y, MULTIPLOT
    %% from pavgnames
    %% where machine like 'i10pc135' and datatype like 'uint64' and size >= 2^13
    %% and algo not like 'ps4oparallel' and algo not like 'tbbparallelsort' and algo not like 'mcstlmwm'
    %% order by titleorder, x
    \addplot+[on layer=foreground, mark layer=foreground] coordinates { (8192,100.558) (16384,82.862) (32768,89.8879) (65536,77.8785) (131072,69.5082) (262144,272.842) (524288,91.485) (1048576,51.7348) (2097152,16.214) (4194304,8.19541) (8388608,4.83204) (16777216,3.0039) (33554432,1.48097) (67108864,1.05922) (1.34218e+08,0.796086) (2.68435e+08,0.65567) (5.36871e+08,0.542512) (1.07374e+09,0.497786) (2.14748e+09,0.491777) (4.29497e+09,0.494022) (8.58993e+09,0.48621) (1.71799e+10,0.48513) };
    \addlegendentry{\compiparassssort};
    \addplot coordinates { (8192,259.368) (16384,213.431) (32768,73.9721) (65536,39.6091) (131072,18.9583) (262144,10.1523) (524288,6.53583) (1048576,4.61542) (2097152,3.49755) (4194304,2.86033) (8388608,2.13747) (16777216,1.70481) (33554432,1.52131) (67108864,1.38695) (1.34218e+08,1.21648) (2.68435e+08,1.10231) (5.36871e+08,1.18071) (1.07374e+09,1.12517) (2.14748e+09,1.20696) (4.29497e+09,1.12074) (8.58993e+09,1.20108) (1.71799e+10,1.12964) };
    \addlegendentry{\compppbbs};
    \addplot coordinates { (8192,5548.01) (16384,2609.65) (32768,1226.88) (65536,565.731) (131072,292.886) (262144,229) (524288,660.58) (1048576,352.109) (2097152,155.595) (4194304,63.9368) (8388608,31.2002) (16777216,15.7925) (33554432,8.27793) (67108864,4.57803) (1.34218e+08,3.02442) (2.68435e+08,2.25418) (5.36871e+08,1.93896) (1.07374e+09,1.9094) (2.14748e+09,1.93468) (4.29497e+09,1.85803) (8.58993e+09,1.65682) (1.71799e+10,1.82224) };
    \addlegendentry{\comppbalancedsort};
    \addplot coordinates { (8192,230.035) (16384,165.264) (32768,69.5297) (65536,32.6178) (131072,17.6271) (262144,13.963) (524288,6.29661) (1048576,4.03188) (2097152,2.43939) (4194304,2.07587) (8388608,1.71851) (16777216,0.959681) (33554432,0.794795) (67108864,0.915652) (1.34218e+08,0.70411) (2.68435e+08,0.620966) (5.36871e+08,0.624313) (1.07374e+09,0.885172) (2.14748e+09,0.605714) (4.29497e+09,0.700743) (8.58993e+09,0.681866) (1.71799e+10,0.790631) };
    \addlegendentry{\radixppbbr};
    \addplot coordinates { (8192,31146.4) (16384,14711.6) (32768,6726.5) (65536,3213.13) (131072,1495.08) (262144,721.121) (524288,333.272) (1048576,146.75) (2097152,72.6527) (4194304,37.1353) (8388608,17.9602) (16777216,9.28852) (33554432,5.15784) (67108864,3.09346) (1.34218e+08,2.19399) (2.68435e+08,1.86665) (5.36871e+08,1.52808) (1.07374e+09,1.24708) (2.14748e+09,1.20945) (4.29497e+09,1.30143) (8.58993e+09,1.27861) (1.71799e+10,1.0915) };
    \addlegendentry{\radixraduls};
    \addplot coordinates { (8192,581.348) (16384,531.872) (32768,2025.49) (65536,1153.41) (131072,625.82) (262144,314.641) (524288,143.951) (1048576,65.0585) (2097152,30.9133) (4194304,15.7186) (8388608,7.98326) (16777216,4.2173) (33554432,3.04761) (67108864,2.24534) (1.34218e+08,1.77937) (2.68435e+08,1.55028) (5.36871e+08,1.5233) (1.07374e+09,1.45049) (2.14748e+09,1.29741) (4.29497e+09,1.22226) (8.58993e+09,1.24997) (1.71799e+10,1.2341) };
    \addlegendentry{\radixregion};
    \addplot+[on layer=foreground, mark layer=foreground] coordinates { (8192,67.5722) (16384,65.2664) (32768,73.3991) (65536,66.99) (131072,49.8963) (262144,237.906) (524288,69.3588) (1048576,37.1239) (2097152,17.6535) (4194304,10.5414) (8388608,8.41928) (16777216,2.55519) (33554432,1.33667) (67108864,0.929411) (1.34218e+08,0.667885) (2.68435e+08,0.536817) (5.36871e+08,0.484496) (1.07374e+09,0.453198) (2.14748e+09,0.469834) (4.29497e+09,0.453161) (8.58993e+09,0.407985) (1.71799e+10,0.400633) };
    \addlegendentry{\compiparassrsort};

      \legend{}
      
    % (Relative) Coordinate at top of the first plot
    \coordinate (c1) at (rel axis cs:0,1);
      
      \nextgroupplot[
      title=\pcamd]
    %% MULTIPLOT(algo|ptitle) select title as ptitle, size as x, 1000000.0 * milli * threads / (8 * size * log(2, size)) as y, MULTIPLOT
    %% from pavgnames
    %% where machine like 'i10pc133' and datatype like 'uint64' and size >= 2^13
    %% and algo not like 'ps4oparallel' and algo not like 'tbbparallelsort' and algo not like 'mcstlmwm'
    %% order by titleorder, x
    \addplot+[on layer=foreground, mark layer=foreground] coordinates { (8192,9.21107) (16384,10.6523) (32768,20.6204) (65536,14.3884) (131072,7.31475) (262144,2.53885) (524288,1.90118) (1048576,1.10824) (2097152,0.750215) (4194304,0.623472) (8388608,0.455158) (16777216,0.46659) (33554432,0.41079) (67108864,0.405667) (1.34218e+08,0.442744) (2.68435e+08,0.467612) (5.36871e+08,0.462489) (1.07374e+09,0.442474) (2.14748e+09,0.434272) (4.29497e+09,0.424741) };
    \addlegendentry{\compiparassssort};
    \addplot coordinates { (8192,7.89384) (16384,5.1107) (32768,3.75834) (65536,2.54592) (131072,1.98698) (262144,1.53511) (524288,1.19784) (1048576,0.887348) (2097152,0.793307) (4194304,0.769014) (8388608,0.832217) (16777216,0.743671) (33554432,0.72027) (67108864,0.688831) (1.34218e+08,0.690336) (2.68435e+08,0.664181) (5.36871e+08,0.662487) (1.07374e+09,0.644752) (2.14748e+09,0.648489) };
    \addlegendentry{\compppbbs};
    \addplot coordinates { (8192,269.7) (16384,168.677) (32768,72.8264) (65536,168.229) (131072,89.9657) (262144,57.0412) (524288,20.709) (1048576,13.6176) (2097152,4.38592) (4194304,2.34262) (8388608,1.28463) (16777216,1.07717) (33554432,1.06264) (67108864,1.10608) (1.34218e+08,1.16016) (2.68435e+08,1.22938) (5.36871e+08,1.28593) (1.07374e+09,1.33899) (2.14748e+09,1.44609) (4.29497e+09,1.48054) };
    \addlegendentry{\comppbalancedsort};
    \addplot coordinates { (8192,5.93031) (16384,3.85277) (32768,2.46876) (65536,1.8176) (131072,1.15574) (262144,0.885609) (524288,0.683664) (1048576,0.591716) (2097152,0.493099) (4194304,0.547112) (8388608,0.563306) (16777216,0.524339) (33554432,0.558912) (67108864,0.666277) (1.34218e+08,0.632071) (2.68435e+08,0.647687) (5.36871e+08,0.632246) (1.07374e+09,0.605324) (2.14748e+09,0.585763) };
    \addlegendentry{\radixppbbr};
    \addplot coordinates { (8192,195.39) (16384,87.215) (32768,38.0288) (65536,18.6988) (131072,9.13094) (262144,4.99075) (524288,2.86704) (1048576,1.70744) (2097152,1.19919) (4194304,0.810808) (8388608,0.627578) (16777216,0.520055) (33554432,0.479368) (67108864,0.4736) (1.34218e+08,0.502371) (2.68435e+08,0.487062) (5.36871e+08,0.50524) (1.07374e+09,0.467859) (2.14748e+09,0.442926) };
    \addlegendentry{\radixraduls};
    \addplot coordinates { (8192,29.3022) (16384,29.5039) (32768,73.6948) (65536,39.0266) (131072,19.9688) (262144,9.88199) (524288,4.9387) (1048576,2.55675) (2097152,1.3665) (4194304,0.910167) (8388608,0.611838) (16777216,0.490682) (33554432,0.617985) (67108864,0.545864) (1.34218e+08,0.528904) (2.68435e+08,0.512003) (5.36871e+08,0.503452) (1.07374e+09,0.493089) (2.14748e+09,0.47194) (4.29497e+09,0.454675) };
    \addlegendentry{\radixregion};
    \addplot+[on layer=foreground, mark layer=foreground] coordinates { (8192,4.74774) (16384,7.90842) (32768,13.4261) (65536,9.93803) (131072,2.36582) (262144,2.41156) (524288,0.806483) (1048576,0.477947) (2097152,0.411248) (4194304,0.37986) (8388608,0.377582) (16777216,0.36934) (33554432,0.371788) (67108864,0.405197) (1.34218e+08,0.486735) (2.68435e+08,0.48199) (5.36871e+08,0.46439) (1.07374e+09,0.455223) (2.14748e+09,0.436261) (4.29497e+09,0.419456) };
    \addlegendentry{\compiparassrsort};

      \legend{}
      
    % (Relative) Coordinate at top of the second plot
    \coordinate (c2) at (rel axis cs:1,1);
      
      \nextgroupplot[
      xlabel={Item count $n$},
      every axis y label/.append style={at=(ticklabel cs:1.1)},
      ylabel={Running time $t/ 8 n\log_2 n$},
      y unit=ns,
      title=\pcinteltwo]
    %% MULTIPLOT(algo|ptitle) select title as ptitle, size as x, 1000000.0 * milli * threads / (8 * size * log(2, size)) as y, MULTIPLOT
    %% from pavgnames
    %% where machine like 'i10pc132' and datatype like 'uint64' and size >= 2^13
    %% and algo not like 'ps4oparallel' and algo not like 'tbbparallelsort' and algo not like 'mcstlmwm'
    %% order by titleorder, x
    \addplot+[on layer=foreground, mark layer=foreground] coordinates { (8192,25.4084) (16384,24.2976) (32768,22.4278) (65536,21.3273) (131072,12.5926) (262144,6.0945) (524288,5.32001) (1048576,2.89062) (2097152,1.71489) (4194304,1.11993) (8388608,0.884531) (16777216,0.742201) (33554432,0.6418) (67108864,0.581225) (1.34218e+08,0.548394) (2.68435e+08,0.525091) (5.36871e+08,0.506375) (1.07374e+09,0.499082) (2.14748e+09,0.482917) (4.29497e+09,0.476268) (8.58993e+09,0.47384) (1.71799e+10,0.476959) };
    \addlegendentry{\compiparassssort};
    \addplot coordinates { (8192,46.8749) (16384,21.1502) (32768,10.1287) (65536,5.58344) (131072,3.71409) (262144,2.93105) (524288,1.94029) (1048576,1.73572) (2097152,1.32606) (4194304,1.28544) (8388608,1.24943) (16777216,1.18535) (33554432,1.16125) (67108864,1.12743) (1.34218e+08,1.11973) (2.68435e+08,1.09923) (5.36871e+08,1.12622) (1.07374e+09,1.09564) (2.14748e+09,1.07855) (4.29497e+09,1.07189) (8.58993e+09,1.06305) (1.71799e+10,1.05425) };
    \addlegendentry{\compppbbs};
    \addplot coordinates { (8192,809.962) (16384,377.693) (32768,194.019) (65536,109.623) (131072,266.791) (262144,149.589) (524288,74.0144) (1048576,36.863) (2097152,16.4857) (4194304,6.06651) (8388608,3.94646) (16777216,2.08507) (33554432,1.59031) (67108864,1.40697) (1.34218e+08,1.30961) (2.68435e+08,1.26544) (5.36871e+08,1.23261) (1.07374e+09,1.31013) (2.14748e+09,1.27507) (4.29497e+09,1.25768) (8.58993e+09,1.25457) (1.71799e+10,1.32572) };
    \addlegendentry{\comppbalancedsort};
    \addplot coordinates { (8192,26.0697) (16384,16.8355) (32768,8.69438) (65536,5.85212) (131072,3.1308) (262144,2.62779) (524288,1.49794) (1048576,1.22113) (2097152,0.809273) (4194304,0.753866) (8388608,0.787527) (16777216,0.655455) (33554432,0.683064) (67108864,0.715723) (1.34218e+08,0.687151) (2.68435e+08,0.684219) (5.36871e+08,0.702349) (1.07374e+09,0.694161) (2.14748e+09,0.685463) (4.29497e+09,0.672939) (8.58993e+09,0.666009) (1.71799e+10,0.641826) };
    \addlegendentry{\radixppbbr};
    \addplot coordinates { (8192,2645.06) (16384,1227.98) (32768,581.772) (65536,271.002) (131072,128.867) (262144,61.3569) (524288,29.727) (1048576,14.4601) (2097152,7.28295) (4194304,4.1474) (8388608,2.25042) (16777216,1.29256) (33554432,0.822415) (67108864,0.61777) (1.34218e+08,0.515045) (2.68435e+08,0.470984) (5.36871e+08,0.43937) (1.07374e+09,0.40449) (2.14748e+09,0.371291) (4.29497e+09,0.371033) (8.58993e+09,0.379706) (1.71799e+10,0.388189) };
    \addlegendentry{\radixraduls};
    \addplot coordinates { (8192,177.15) (16384,184.622) (32768,414.695) (65536,226.606) (131072,115.728) (262144,57.6702) (524288,28.3663) (1048576,13.9168) (2097152,6.93938) (4194304,3.58016) (8388608,2.05628) (16777216,1.3475) (33554432,0.98566) (67108864,0.716136) (1.34218e+08,0.565491) (2.68435e+08,0.488552) (5.36871e+08,0.655353) (1.07374e+09,0.589949) (2.14748e+09,0.4657) (4.29497e+09,0.416) (8.58993e+09,0.422256) (1.71799e+10,0.456845) };
    \addlegendentry{\radixregion};
    \addplot+[on layer=foreground, mark layer=foreground] coordinates { (8192,44.5742) (16384,33.4266) (32768,29.3781) (65536,43.7594) (131072,20.7666) (262144,10.2545) (524288,4.98714) (1048576,2.63115) (2097152,1.56789) (4194304,1.02331) (8388608,0.830675) (16777216,0.665277) (33554432,0.5307) (67108864,0.476613) (1.34218e+08,0.470372) (2.68435e+08,0.453902) (5.36871e+08,0.442587) (1.07374e+09,0.445165) (2.14748e+09,0.461455) (4.29497e+09,0.445354) (8.58993e+09,0.420887) (1.71799e+10,0.402193) };
    \addlegendentry{\compiparassrsort};

      \legend{}

      \nextgroupplot[
      xlabel={Item count $n$},
      title=\pcintelfour,
      every legend/.append style={at=(ticklabel cs:1.1)},
      legend style={at={($(0,0)+(1cm,1cm)$)},legend columns=5,fill=none,draw=black,anchor=center,align=center},
      legend to name=legendplotp
      ]
    %% MULTIPLOT(algo|ptitle) select title as ptitle, size as x, 1000000.0 * milli * threads / (8 * size * log(2, size)) as y, MULTIPLOT
    %% from pavgnames
    %% where machine like 'i10pc136' and datatype like 'uint64' and size >= 2^13
    %% and algo not like 'ps4oparallel' and algo not like 'tbbparallelsort' and algo not like 'mcstlmwm'
    %% order by titleorder, x
    \addplot+[on layer=foreground, mark layer=foreground] coordinates { (8192,36.044) (16384,32.4001) (32768,29.9261) (65536,26.1731) (131072,90.5515) (262144,48.1624) (524288,24.3135) (1048576,11.8765) (2097152,5.97586) (4194304,3.33518) (8388608,1.96277) (16777216,1.26028) (33554432,0.815912) (67108864,0.625805) (1.34218e+08,0.540967) (2.68435e+08,0.500081) (5.36871e+08,0.467528) (1.07374e+09,0.452329) (2.14748e+09,0.430276) (4.29497e+09,0.424179) (8.58993e+09,0.417892) (1.71799e+10,0.416613) };
    \addlegendentry{\compiparassssort};
    \addplot coordinates { (8192,67.4571) (16384,38.6036) (32768,19.0275) (65536,10.0741) (131072,6.91886) (262144,4.72144) (524288,3.24564) (1048576,2.48397) (2097152,2.43418) (4194304,1.73837) (8388608,1.48853) (16777216,1.25697) (33554432,1.05542) (67108864,0.952569) (1.34218e+08,0.907264) (2.68435e+08,0.868348) (5.36871e+08,0.856286) (1.07374e+09,0.833487) (2.14748e+09,0.83366) (4.29497e+09,0.827392) (8.58993e+09,0.824717) (1.71799e+10,0.827187) };
    \addlegendentry{\compppbbs};
    \addplot coordinates { (8192,2554.86) (16384,1190.3) (32768,538.14) (65536,259.738) (131072,125.98) (262144,408.964) (524288,218.289) (1048576,95.0953) (2097152,46.9502) (4194304,27.2228) (8388608,11.2783) (16777216,5.29861) (33554432,3.02936) (67108864,1.7904) (1.34218e+08,1.40502) (2.68435e+08,1.29377) (5.36871e+08,1.20868) (1.07374e+09,1.17937) (2.14748e+09,1.22924) (4.29497e+09,1.19382) (8.58993e+09,1.20298) (1.71799e+10,1.23756) };
    \addlegendentry{\comppbalancedsort};
    \addplot coordinates { (8192,52.9688) (16384,34.293) (32768,15.9934) (65536,9.42988) (131072,5.36045) (262144,5.80338) (524288,2.39102) (1048576,2.02866) (2097152,1.53177) (4194304,1.37644) (8388608,1.17545) (16777216,0.869737) (33554432,0.621215) (67108864,0.635606) (1.34218e+08,0.594969) (2.68435e+08,0.58366) (5.36871e+08,0.589864) (1.07374e+09,0.551571) (2.14748e+09,0.518685) (4.29497e+09,0.522189) (8.58993e+09,0.506355) (1.71799e+10,0.500243) };
    \addlegendentry{\radixppbbr};
    \addplot coordinates { (8192,6268.88) (16384,2890.54) (32768,1368.26) (65536,659.937) (131072,318.765) (262144,154.037) (524288,74.7112) (1048576,36.8155) (2097152,17.8391) (4194304,9.81669) (8388608,5.32285) (16777216,2.79481) (33554432,1.58304) (67108864,1.0004) (1.34218e+08,0.709608) (2.68435e+08,0.556756) (5.36871e+08,0.505768) (1.07374e+09,0.43976) (2.14748e+09,0.399433) (4.29497e+09,0.397535) (8.58993e+09,0.403518) (1.71799e+10,0.399576) };
    \addlegendentry{\radixraduls};
    \addplot coordinates { (8192,163.942) (16384,155.319) (32768,703.557) (65536,395.945) (131072,208.422) (262144,103.12) (524288,52.5127) (1048576,26.7517) (2097152,13.2696) (4194304,6.62146) (8388608,3.31544) (16777216,2.09448) (33554432,1.54502) (67108864,0.964909) (1.34218e+08,0.78287) (2.68435e+08,0.659037) (5.36871e+08,0.615859) (1.07374e+09,0.549614) (2.14748e+09,0.474463) (4.29497e+09,0.386302) (8.58993e+09,0.405818) (1.71799e+10,0.461207) };
    \addlegendentry{\radixregion};
    \addplot+[on layer=foreground, mark layer=foreground] coordinates { (8192,29.9185) (16384,28.3039) (32768,30.3215) (65536,22.5025) (131072,114.676) (262144,45.2224) (524288,23.1386) (1048576,11.7152) (2097152,5.60484) (4194304,3.19666) (8388608,1.89331) (16777216,1.24619) (33554432,0.706356) (67108864,0.516161) (1.34218e+08,0.49045) (2.68435e+08,0.448126) (5.36871e+08,0.443605) (1.07374e+09,0.440705) (2.14748e+09,0.438575) (4.29497e+09,0.407635) (8.58993e+09,0.383502) (1.71799e+10,0.386269) };
    \addlegendentry{\compiparassrsort};

    \end{groupplot}
    \coordinate (c3) at ($(c1)!.5!(c2)$);
    \node[below] at (c3 |- current bounding box.south)
    {\pgfplotslegendfromname{legendplotp}};
  \end{tikzpicture}
  \caption{
    Running times of parallel algorithms sorting \ulong values with input distribution \distuniform executed on different machines.
  }
  \label{fig:rt rand par}
\end{figure}


\begin{figure}[tbp]
  %% SQL
%% drop view if exists p CASCADE;
%% create view p as
%% select * from pradixalgoswithips4oml
%% union
%% select * from pcomparisonalgos

%% SQL
%% drop view if exists palgos1 CASCADE;
%% create view palgos1 as
%% select avgparallel.* from avgparallel inner join p
%% on avgparallel.algo = p.algo

%% SQL
%% drop view if exists palgos2 CASCADE;
%% create view palgos2 as
%% select * from palgos1 natural join pfast

%% SQL
%% drop view if exists palgos CASCADE;
%% create view palgos as
%% select machine, gen, datatype, algo, parallel, threads, vector, size, meminterleaved, copyback, milli
%% from palgos2

%% SQL
%% drop view if exists pavg CASCADE;
%% create view pavg as
%% select machine, algo, size, datatype, gen, threads, AVG(milli) as milli
%% from palgos
%% group by machine, algo, size, datatype, gen, threads


%% SQL
%% drop table if exists undisplayed CASCADE;
%% create table undisplayed(
%% machine character varying,
%% algo character varying,
%% size            bigint,
%% datatype character varying,
%% gen character varying,
%% threads            bigint,
%% milli  double precision
%% )

%% SQL
%% insert into undisplayed (machine, algo, datatype, gen, threads, size, milli)
%% values
%% ('i10pc136', 'pbbsradixsort', 'double', 'random', 1, -1, 0.5*2^19*19*8/1000000),
%% ('i10pc136', 'pbbsradixsort', 'double', 'random', 1, -2, 0.3*2^19*19*8/1000000),
%% ('i10pc136', 'pbbsradixsort', 'double', 'random', 1, -3, 0.1*2^19*19*8/1000000),
%% ('i10pc136', 'pbbsradixsort', 'uint64', 'almostsorted', 1, -1, 0.1*2^19.5*19.5*8/1000000),
%% ('i10pc136', 'pbbsradixsort', 'uint64', 'almostsorted', 1, -2, 0.1*2^20.5*20.5*8/1000000),
%% ('i10pc136', 'pbbsradixsort', 'uint64', 'almostsorted', 1, -3, 0.1*2^21.5*21.5*8/1000000),
%% ('i10pc136', 'pbbsradixsort', 'uint64', 'rootdupls', 1, -1, 0.3*2^19.5*19.5*8/1000000),
%% ('i10pc136', 'pbbsradixsort', 'uint64', 'rootdupls', 1, -2, 0.3*2^20.5*20.5*8/1000000),
%% ('i10pc136', 'pbbsradixsort', 'uint64', 'rootdupls', 1, -3, 0.3*2^21.5*21.5*8/1000000),
%% ('i10pc136', 'pbbsradixsort', 'uint64', 'twicedupes', 1, -1, 0.3*2^19.5*19.5*8/1000000),
%% ('i10pc136', 'pbbsradixsort', 'uint64', 'twicedupes', 1, -2, 0.3*2^20.5*20.5*8/1000000),
%% ('i10pc136', 'pbbsradixsort', 'uint64', 'twicedupes', 1, -3, 0.3*2^21.5*21.5*8/1000000),
%% ('i10pc136', 'pbbsradixsort', 'uint64', 'zipf', 1, -1, 0.3*2^19.5*19.5*8/1000000),
%% ('i10pc136', 'pbbsradixsort', 'uint64', 'zipf', 1, -2, 0.3*2^20.5*20.5*8/1000000),
%% ('i10pc136', 'pbbsradixsort', 'uint64', 'zipf', 1, -3, 0.3*2^21.5*21.5*8/1000000),
%% ('i10pc136', 'raduls', 'double', 'random', 1, -1, 0.5*2^19.5*19.5*8/1000000),
%% ('i10pc136', 'raduls', 'double', 'random', 1, -2, 0.3*2^19.5*19.5*8/1000000),
%% ('i10pc136', 'raduls', 'double', 'random', 1, -3, 0.1*2^19.5*19.5*8/1000000),
%% ('i10pc136', 'regionsort', 'double', 'random', 1, -1, 0.5*2^20*20*8/1000000),
%% ('i10pc136', 'regionsort', 'double', 'random', 1, -2, 0.3*2^20*20*8/1000000),
%% ('i10pc136', 'regionsort', 'double', 'random', 1, -3, 0.1*2^20*20*8/1000000),
%% ('i10pc136', 'pbbsradixsort', 'byte', 'byte', 1, -1, 0.5*2^15.5*15.5*100/1000000),
%% ('i10pc136', 'pbbsradixsort', 'byte', 'byte', 1, -2, 0.3*2^15.5*15.5*100/1000000),
%% ('i10pc136', 'pbbsradixsort', 'byte', 'byte', 1, -3, 0.1*2^15.5*15.5*100/1000000),
%% ('i10pc136', 'raduls', 'byte', 'byte', 1, -1, 0.5*2^16*16*100/1000000),
%% ('i10pc136', 'raduls', 'byte', 'byte', 1, -2, 0.3*2^16*16*100/1000000),
%% ('i10pc136', 'raduls', 'byte', 'byte', 1, -3, 0.1*2^16*16*100/1000000),
%% ('i10pc136', 'regionsort', 'byte', 'byte', 1, -1, 0.5*2^16.5*16.5*100/1000000),
%% ('i10pc136', 'regionsort', 'byte', 'byte', 1, -2, 0.3*2^16.5*16.5*100/1000000),
%% ('i10pc136', 'regionsort', 'byte', 'byte', 1, -3, 0.1*2^16.5*16.5*100/1000000)

%% SQL
%% drop view if exists pavg1 CASCADE;
%% create view pavg1 as
%% select * from undisplayed union select * from pavg

%% SQL
%% drop view if exists pavgnames CASCADE;
%% create view pavgnames as
%% select pavg1.*, titles.title, titles.titleorder, datatypesizes.dsize from pavg1
%% inner join titles
%% on titles.algo like pavg1.algo
%% inner join datatypesizes
%% on pavg1.datatype = datatypesizes.datatype

  \begin{tikzpicture}
    \begin{groupplot}[
      group style={
        group size=2 by 4,
        vertical sep=1.2cm,
      },
      width=0.5\textwidth,
      height=0.22\textheight,
      xmode=log,
      log base x=2,
      plotstyleparallel,
      xmajorgrids=true,
      ymajorgrids=true,
      ymin=0,
      ymax=1.5,
      xmin=2^18,
      xmax=2^34,
      xtick={2^18, 2^22, 2^26, 2^30, 2^34},
      restrict y to domain=-1:30,
      every axis title/.append style={yshift=-0.2cm},
      ]
      
      
      \nextgroupplot[
      title=\distuniform \double]
      %% MULTIPLOT(algo|ptitle) select title as ptitle, size as x, 1000000.0 * milli * threads / (dsize * size * log(2, size)) as y, MULTIPLOT
      %% from pavgnames
      %% where algo not like 'aspasparallel' and algo not like 'mcstlmwm' and algo not like 'tbbparallelsort' and  algo not like 'ps4oparallel' and machine like 'i10pc136' and datatype like 'double' and size >= 2^13
      %% and (algo not like 'tbbparallelsort' or 1000000.0 * milli * threads / (dsize * size * log(2, size)) <= 2)
      %% and gen like 'random'
      %% order by titleorder, x    
      \addplot coordinates { (8192,43.6426) (16384,39.489) (32768,36.7912) (65536,34.5816) (131072,106.615) (262144,50.3939) (524288,25.8974) (1048576,12.4511) (2097152,6.53255) (4194304,3.40339) (8388608,2.13182) (16777216,1.31154) (33554432,0.871913) (67108864,0.687136) (1.34218e+08,0.591505) (2.68435e+08,0.543183) (5.36871e+08,0.516928) (1.07374e+09,0.496874) (2.14748e+09,0.479056) (4.29497e+09,0.467071) (8.58993e+09,0.459931) (1.71799e+10,0.460476) };
      \addlegendentry{\compiparassssort};
      \addplot coordinates { (8192,64.9241) (16384,80.0873) (32768,24.1362) (65536,10.6159) (131072,11.5663) (262144,4.99093) (524288,3.59981) (1048576,2.84658) (2097152,2.28737) (4194304,2.00722) (8388608,1.67594) (16777216,1.43034) (33554432,1.25308) (67108864,1.14817) (1.34218e+08,1.10521) (2.68435e+08,1.06381) (5.36871e+08,1.0494) (1.07374e+09,1.03326) (2.14748e+09,1.03119) (4.29497e+09,1.02317) (8.58993e+09,1.01742) (1.71799e+10,1.01735) };
      \addlegendentry{\compppbbs};
      \addplot coordinates { (8192,2350.79) (16384,1110.26) (32768,556.991) (65536,265.664) (131072,108.458) (262144,445.169) (524288,195.106) (1048576,108.082) (2097152,49.2104) (4194304,25.2979) (8388608,11.3485) (16777216,5.29649) (33554432,3.22252) (67108864,1.96558) (1.34218e+08,1.64614) (2.68435e+08,1.53304) (5.36871e+08,1.45518) (1.07374e+09,1.47223) (2.14748e+09,1.39688) (4.29497e+09,1.48431) (8.58993e+09,1.40863) (1.71799e+10,1.40581) };
      \addlegendentry{\comppbalancedsort};

      \legend{}
      
      % (Relative) Coordinate at top of the first plot
      \coordinate (c1) at (rel axis cs:0,1);
      
      \nextgroupplot[
      title=\distalmostsorted \ulong]
      %% MULTIPLOT(algo|ptitle) select title as ptitle, size as x, 1000000.0 * milli * threads / (dsize * size * log(2, size)) as y, MULTIPLOT
      %% from pavgnames
      %% where algo not like 'mcstlmwm' and algo not like 'tbbparallelsort' and  algo not like 'ps4oparallel' and machine like 'i10pc136' and datatype like 'uint64' and size >= 2^13
      %% and gen like 'almostsorted'
      %% order by titleorder, x
      \addplot coordinates { (8192,24.3704) (16384,23.3746) (32768,21.1469) (65536,19.4153) (131072,91.4134) (262144,50.2235) (524288,14.3513) (1048576,7.80931) (2097152,3.47093) (4194304,2.204) (8388608,1.47843) (16777216,0.867296) (33554432,0.556735) (67108864,0.402409) (1.34218e+08,0.368458) (2.68435e+08,0.339532) (5.36871e+08,0.318304) (1.07374e+09,0.288189) (2.14748e+09,0.273013) (4.29497e+09,0.262226) (8.58993e+09,0.258194) (1.71799e+10,0.257936) };
      \addlegendentry{\compiparassssort};
      \addplot coordinates { (8192,42.2081) (16384,58.5947) (32768,18.905) (65536,8.6873) (131072,5.91688) (262144,3.56202) (524288,2.33961) (1048576,1.72956) (2097152,1.26236) (4194304,1.15146) (8388608,1.13276) (16777216,0.856108) (33554432,0.655217) (67108864,0.535787) (1.34218e+08,0.491445) (2.68435e+08,0.457702) (5.36871e+08,0.438066) (1.07374e+09,0.416935) (2.14748e+09,0.415036) (4.29497e+09,0.411009) (8.58993e+09,0.416349) (1.71799e+10,0.417098) };
      \addlegendentry{\compppbbs};
      \addplot coordinates { (8192,2887.82) (16384,1357.53) (32768,540.015) (65536,283.714) (131072,137.119) (262144,364.947) (524288,202.961) (1048576,94.2005) (2097152,49.6927) (4194304,22.7695) (8388608,10.1715) (16777216,5.52582) (33554432,2.93855) (67108864,1.83027) (1.34218e+08,1.27265) (2.68435e+08,0.991044) (5.36871e+08,0.784492) (1.07374e+09,0.591159) (2.14748e+09,0.628118) (4.29497e+09,0.629134) (8.58993e+09,0.657914) (1.71799e+10,0.652994) };
      \addlegendentry{\comppbalancedsort};
      \addplot coordinates { (8192,75.7088) (16384,70.3621) (32768,37.4777) (65536,26.4634) (131072,16.6065) (262144,12.9293) (524288,10.2371) (1048576,8.43218) (2097152,6.56278) (4194304,5.08272) (8388608,4.97772) (16777216,4.52476) (33554432,3.56616) (67108864,3.07309) (1.34218e+08,2.79083) (2.68435e+08,3.00535) (5.36871e+08,2.84432) (1.07374e+09,2.67677) (2.14748e+09,2.22607) (4.29497e+09,2.84126) (8.58993e+09,2.70928) (1.71799e+10,2.69563) };
      \addlegendentry{\radixppbbr};
      \addplot coordinates { (8192,41348.5) (16384,19276.5) (32768,8956.2) (65536,4228.43) (131072,2281.75) (262144,1341.87) (524288,878.983) (1048576,656.853) (2097152,535.315) (4194304,43.2541) (8388608,21.165) (16777216,10.9437) (33554432,6.09492) (67108864,4.08889) (1.34218e+08,3.03763) (2.68435e+08,2.52381) (5.36871e+08,2.24831) (1.07374e+09,0.876879) (2.14748e+09,0.785027) (4.29497e+09,0.74045) (8.58993e+09,0.707269) (1.71799e+10,0.681999) };
      \addlegendentry{\radixraduls};
      \addplot coordinates { (8192,99.5475) (16384,67.2686) (32768,80.7869) (65536,60.0557) (131072,15.9703) (262144,8.57237) (524288,5.67021) (1048576,3.2356) (2097152,2.59418) (4194304,1.61524) (8388608,0.931916) (16777216,0.938441) (33554432,0.584401) (67108864,0.330145) (1.34218e+08,0.393876) (2.68435e+08,0.305581) (5.36871e+08,0.265231) (1.07374e+09,0.242757) (2.14748e+09,0.226755) (4.29497e+09,0.220803) (8.58993e+09,0.304453) (1.71799e+10,0.31874) };
      \addlegendentry{\radixregion};
      \addplot coordinates { (8192,19.9827) (16384,17.1428) (32768,13.4486) (65536,11.5319) (131072,90.4021) (262144,39.3591) (524288,20.2389) (1048576,11.4903) (2097152,7.34105) (4194304,3.21765) (8388608,1.44552) (16777216,1.01926) (33554432,0.855388) (67108864,0.547379) (1.34218e+08,0.412638) (2.68435e+08,0.341552) (5.36871e+08,0.301665) (1.07374e+09,0.280671) (2.14748e+09,0.281432) (4.29497e+09,0.258652) (8.58993e+09,0.420369) (1.71799e+10,0.368404) };
      \addlegendentry{\compiparassrsort};

      \legend{}
      
      % (Relative) Coordinate at top of the second plot
      \coordinate (c2) at (rel axis cs:1,1);
      
      \nextgroupplot[
      y unit=ns,
      title=\distduplicatesroot \ulong]
      %% MULTIPLOT(algo|ptitle) select title as ptitle, size as x, 1000000.0 * milli * threads / (dsize * size * log(2, size)) as y, MULTIPLOT
      %% from pavgnames
      %% where algo not like 'mcstlmwm' and algo not like 'tbbparallelsort' and  algo not like 'ps4oparallel' and machine like 'i10pc136' and datatype like 'uint64' and size >= 2^13
      %% and gen like 'rootdupls'
      %% and (algo not like 'tbbparallelsort' or 1000000.0 * milli * threads / (dsize * size * log(2, size)) <= 2)
      %% order by titleorder, x
      \addplot coordinates { (8192,20.16) (16384,18.5269) (32768,15.2467) (65536,15.5449) (131072,88.7386) (262144,49.2668) (524288,32.9261) (1048576,10.5691) (2097152,5.47893) (4194304,2.78113) (8388608,1.7174) (16777216,1.10221) (33554432,0.655242) (67108864,0.446698) (1.34218e+08,0.398181) (2.68435e+08,0.364723) (5.36871e+08,0.341459) (1.07374e+09,0.332147) (2.14748e+09,0.317779) (4.29497e+09,0.312481) (8.58993e+09,0.331798) (1.71799e+10,0.335706) };
      \addlegendentry{\compiparassssort};
      \addplot coordinates { (8192,41.4271) (16384,44.167) (32768,26.325) (65536,7.84515) (131072,5.72561) (262144,3.34929) (524288,2.39939) (1048576,1.75149) (2097152,1.53872) (4194304,1.1438) (8388608,1.16207) (16777216,0.952045) (33554432,0.669097) (67108864,0.554029) (1.34218e+08,0.526744) (2.68435e+08,0.487775) (5.36871e+08,0.474155) (1.07374e+09,0.455989) (2.14748e+09,0.46855) (4.29497e+09,0.464959) (8.58993e+09,0.479698) (1.71799e+10,0.469748) };
      \addlegendentry{\compppbbs};
      \addplot coordinates { (8192,2938.86) (16384,1390.64) (32768,592.223) (65536,278.164) (131072,156.577) (262144,699.405) (524288,409.701) (1048576,185.496) (2097152,72.6852) (4194304,43.5766) (8388608,13.5658) (16777216,11.0901) (33554432,3.1417) (67108864,3.07444) (1.34218e+08,1.40424) (2.68435e+08,1.39439) (5.36871e+08,0.986438) (1.07374e+09,1.08251) (2.14748e+09,0.91049) (4.29497e+09,1.08496) (8.58993e+09,0.930002) (1.71799e+10,1.08879) };
      \addlegendentry{\comppbalancedsort};
      \addplot coordinates { (8192,74.7315) (16384,76.8779) (32768,56.2478) (65536,28.7731) (131072,18.1869) (262144,15.7302) (524288,12.0771) (1048576,9.04944) (2097152,7.20024) (4194304,5.77612) (8388608,6.49069) (16777216,4.73663) (33554432,3.92704) (67108864,3.46687) (1.34218e+08,3.682) (2.68435e+08,3.28155) (5.36871e+08,3.07325) (1.07374e+09,3.11717) (2.14748e+09,2.81368) (4.29497e+09,3.80502) (8.58993e+09,3.67564) };
      \addlegendentry{\radixppbbr};
      \addplot coordinates { (8192,46250.6) (16384,21434.6) (32768,10023.5) (65536,4713.86) (131072,2461.05) (262144,1160.74) (524288,602.193) (1048576,312.973) (2097152,171.981) (4194304,97.0051) (8388608,61.0212) (16777216,36.394) (33554432,22.4602) (67108864,14.5274) (1.34218e+08,9.6434) (2.68435e+08,1.42145) (5.36871e+08,1.11425) (1.07374e+09,0.95126) (2.14748e+09,0.862378) (4.29497e+09,0.806107) (8.58993e+09,0.769775) (1.71799e+10,0.737449) };
      \addlegendentry{\radixraduls};
      \addplot coordinates { (8192,76.7218) (16384,40.3611) (32768,409.842) (65536,52.6369) (131072,36.809) (262144,11.8777) (524288,8.90956) (1048576,4.03948) (2097152,2.89121) (4194304,1.53849) (8388608,1.07067) (16777216,0.970431) (33554432,0.767156) (67108864,0.577023) (1.34218e+08,0.514802) (2.68435e+08,0.475116) (5.36871e+08,0.449524) (1.07374e+09,0.432695) (2.14748e+09,0.404593) (4.29497e+09,0.401994) (8.58993e+09,0.475641) (1.71799e+10,0.539893) };
      \addlegendentry{\radixregion};
      \addplot coordinates { (8192,12.2855) (16384,8.82932) (32768,7.45273) (65536,9.31701) (131072,131.307) (262144,63.401) (524288,30.0146) (1048576,12.7722) (2097152,7.03959) (4194304,3.99448) (8388608,2.54715) (16777216,1.46041) (33554432,0.914796) (67108864,0.664572) (1.34218e+08,0.425132) (2.68435e+08,0.390118) (5.36871e+08,0.351739) (1.07374e+09,0.325274) (2.14748e+09,0.324879) (4.29497e+09,0.500906) (8.58993e+09,0.475262) (1.71799e+10,0.461731) };
      \addlegendentry{\compiparassrsort};

      \legend{}

      \nextgroupplot[
      title=\distduplicatestwice \ulong,
      ]
      %% MULTIPLOT(algo|ptitle) select title as ptitle, size as x, 1000000.0 * milli * threads / (dsize * size * log(2, size)) as y, MULTIPLOT
      %% from pavgnames
      %% where algo not like 'mcstlmwm' and algo not like 'tbbparallelsort' and  algo not like 'ps4oparallel' and machine like 'i10pc136' and datatype like 'uint64' and size >= 2^13
      %% and (algo not like 'tbbparallelsort' or 1000000.0 * milli * threads / (dsize * size * log(2, size)) <= 2)
      %% and gen like 'twicedupes'
      %% order by titleorder, x
      \addplot coordinates { (8192,33.1428) (16384,28.4567) (32768,29.2237) (65536,26.5533) (131072,110.158) (262144,54.2595) (524288,22.8573) (1048576,13.3921) (2097152,6.62881) (4194304,3.48176) (8388608,1.90946) (16777216,1.22108) (33554432,0.81175) (67108864,0.602592) (1.34218e+08,0.536397) (2.68435e+08,0.488872) (5.36871e+08,0.458856) (1.07374e+09,0.43904) (2.14748e+09,0.420888) (4.29497e+09,0.413464) (8.58993e+09,0.40699) (1.71799e+10,0.410319) };
      \addlegendentry{\compiparassssort};
      \addplot coordinates { (8192,60.0757) (16384,40.5349) (32768,18.0552) (65536,9.80608) (131072,6.53497) (262144,4.43915) (524288,3.04367) (1048576,3.19074) (2097152,2.14097) (4194304,1.72013) (8388608,1.42642) (16777216,1.20335) (33554432,1.01016) (67108864,0.89505) (1.34218e+08,0.84898) (2.68435e+08,0.812258) (5.36871e+08,0.799331) (1.07374e+09,0.778536) (2.14748e+09,0.777067) (4.29497e+09,0.77683) (8.58993e+09,0.778236) (1.71799e+10,0.774074) };
      \addlegendentry{\compppbbs};
      \addplot coordinates { (8192,2639.65) (16384,1198.12) (32768,462.615) (65536,269.252) (131072,134.313) (262144,368.891) (524288,199.591) (1048576,104.03) (2097152,44.8785) (4194304,24.622) (8388608,10.8496) (16777216,5.77974) (33554432,3.34592) (67108864,1.97083) (1.34218e+08,1.55036) (2.68435e+08,1.39053) (5.36871e+08,1.33653) (1.07374e+09,1.19705) (2.14748e+09,1.13516) (4.29497e+09,1.16085) (8.58993e+09,1.2318) (1.71799e+10,1.28331) };
      \addlegendentry{\comppbalancedsort};
      \addplot coordinates { (8192,104.117) (16384,78.8034) (32768,50.168) (65536,37.6939) (131072,26.2858) (262144,21.9138) (524288,16.9551) (1048576,13.6647) (2097152,9.54867) (4194304,9.2872) (8388608,8.34604) (16777216,7.80334) (33554432,6.38079) (67108864,5.59283) (1.34218e+08,4.9688) (2.68435e+08,7.21257) (5.36871e+08,6.03146) (1.07374e+09,5.35026) (2.14748e+09,3.66989) (4.29497e+09,3.28734) (8.58993e+09,2.99416) };
      \addlegendentry{\radixppbbr};
      \addplot coordinates { (8192,41573.7) (16384,19335.9) (32768,9054.56) (65536,4228.53) (131072,2284.9) (262144,1354.67) (524288,902.059) (1048576,670.515) (2097152,550.184) (4194304,124.765) (8388608,21.2193) (16777216,10.9997) (33554432,6.07093) (67108864,4.02803) (1.34218e+08,3.00102) (2.68435e+08,2.50959) (5.36871e+08,2.23436) (1.07374e+09,1.07881) (2.14748e+09,0.784265) (4.29497e+09,0.741191) (8.58993e+09,0.708437) (1.71799e+10,0.683977) };
      \addlegendentry{\radixraduls};
      \addplot coordinates { (8192,87.2533) (16384,53.6899) (32768,422.744) (65536,399.685) (131072,55.9599) (262144,27.0544) (524288,13.581) (1048576,7.21252) (2097152,5.52527) (4194304,2.9482) (8388608,1.81295) (16777216,2.09879) (33554432,1.02965) (67108864,0.691013) (1.34218e+08,0.56615) (2.68435e+08,0.499231) (5.36871e+08,0.46604) (1.07374e+09,0.413504) (2.14748e+09,0.398383) (4.29497e+09,0.385903) (8.58993e+09,0.480805) (1.71799e+10,0.558243) };
      \addlegendentry{\radixregion};
      \addplot coordinates { (8192,18.7699) (16384,15.2498) (32768,13.3622) (65536,12.223) (131072,127.793) (262144,68.2988) (524288,33.2871) (1048576,20.8334) (2097152,10.2345) (4194304,4.70056) (8388608,1.7763) (16777216,1.08963) (33554432,1.11889) (67108864,0.666862) (1.34218e+08,0.533385) (2.68435e+08,0.48231) (5.36871e+08,0.442289) (1.07374e+09,0.408003) (2.14748e+09,0.389057) (4.29497e+09,0.375673) (8.58993e+09,0.522669) (1.71799e+10,0.46633) };
      \addlegendentry{\compiparassrsort};

      \legend{}
      
      \nextgroupplot[
      every axis y label/.append style={at=(ticklabel cs:1.1)},
      ylabel={Running time $t/ Dn \log_2 n$},
      title=\distexpo \ulong]
      %% MULTIPLOT(algo|ptitle) select title as ptitle, size as x, 1000000.0 * milli * threads / (dsize * size * log(2, size)) as y, MULTIPLOT
      %% from pavgnames
      %% where algo not like 'mcstlmwm' and algo not like 'tbbparallelsort' and  algo not like 'ps4oparallel' and machine like 'i10pc136' and datatype like 'uint64' and size >= 2^13
      %% and (algo not like 'tbbparallelsort' or 1000000.0 * milli * threads / (dsize * size * log(2, size)) <= 2)
      %% and gen like 'exponential'
      %% order by titleorder, x
      \addplot coordinates { (8192,30.8999) (16384,26.6938) (32768,23.0684) (65536,20.908) (131072,91.3699) (262144,52.2472) (524288,35.1554) (1048576,17.2039) (2097152,8.15052) (4194304,4.29228) (8388608,2.34532) (16777216,1.43089) (33554432,0.832136) (67108864,0.583596) (1.34218e+08,0.47927) (2.68435e+08,0.421549) (5.36871e+08,0.390156) (1.07374e+09,0.368363) (2.14748e+09,0.351929) (4.29497e+09,0.34546) (8.58993e+09,0.328195) (1.71799e+10,0.327431) };
      \addlegendentry{\compiparassssort};
      \addplot coordinates { (8192,62.6858) (16384,42.2688) (32768,19.2149) (65536,9.51592) (131072,6.48339) (262144,4.17637) (524288,2.85812) (1048576,2.19065) (2097152,1.66694) (4194304,1.46927) (8388608,1.40946) (16777216,1.08847) (33554432,0.872255) (67108864,0.742732) (1.34218e+08,0.709355) (2.68435e+08,0.661186) (5.36871e+08,0.650473) (1.07374e+09,0.620455) (2.14748e+09,0.619785) (4.29497e+09,0.612511) (8.58993e+09,0.608204) (1.71799e+10,0.603059) };
      \addlegendentry{\compppbbs};
      \addplot coordinates { (8192,2456.17) (16384,1089.03) (32768,482.528) (65536,267.821) (131072,164.681) (262144,396.032) (524288,223.704) (1048576,105.756) (2097152,50.054) (4194304,26.5746) (8388608,11.164) (16777216,5.31308) (33554432,3.15943) (67108864,1.94028) (1.34218e+08,1.57004) (2.68435e+08,1.35134) (5.36871e+08,1.24979) (1.07374e+09,1.1064) (2.14748e+09,1.08699) (4.29497e+09,1.06368) (8.58993e+09,1.07465) (1.71799e+10,1.08398) };
      \addlegendentry{\comppbalancedsort};
      \addplot coordinates { (8192,48.4417) (16384,37.313) (32768,47.0732) (65536,10.6254) (131072,6.74278) (262144,4.50733) (524288,2.78575) (1048576,2.23578) (2097152,1.51078) (4194304,1.37532) (8388608,1.19044) (16777216,0.948814) (33554432,0.674547) (67108864,0.620974) (1.34218e+08,0.576242) (2.68435e+08,0.601946) (5.36871e+08,0.647451) (1.07374e+09,0.653416) (2.14748e+09,0.630437) (4.29497e+09,0.617744) (8.58993e+09,0.606212) (1.71799e+10,0.635981) };
      \addlegendentry{\radixppbbr};
      \addplot coordinates { (8192,15855.2) (16384,18386.1) (32768,12872.6) (65536,4215.67) (131072,1759.28) (262144,628.628) (524288,266.568) (1048576,120.043) (2097152,53.4312) (4194304,27.4926) (8388608,12.781) (16777216,6.19072) (33554432,3.32697) (67108864,2.05497) (1.34218e+08,1.39249) (2.68435e+08,0.988671) (5.36871e+08,0.769149) (1.07374e+09,0.662426) (2.14748e+09,0.742732) (4.29497e+09,0.886914) (8.58993e+09,0.847887) (1.71799e+10,0.821202) };
      \addlegendentry{\radixraduls};
      \addplot coordinates { (8192,228.168) (16384,209.827) (32768,642.666) (65536,345.034) (131072,191.25) (262144,98.2888) (524288,50.5779) (1048576,25.2721) (2097152,13.2376) (4194304,6.89917) (8388608,3.49308) (16777216,2.21397) (33554432,1.50816) (67108864,1.00642) (1.34218e+08,0.804259) (2.68435e+08,0.721926) (5.36871e+08,0.649455) (1.07374e+09,0.589132) (2.14748e+09,0.540665) (4.29497e+09,0.444666) (8.58993e+09,0.467781) (1.71799e+10,0.516459) };
      \addlegendentry{\radixregion};
      \addplot coordinates { (8192,32.0172) (16384,29.0868) (32768,27.7164) (65536,22.9701) (131072,153.793) (262144,63.0587) (524288,26.4905) (1048576,14.6238) (2097152,6.95281) (4194304,4.05387) (8388608,2.22447) (16777216,1.40606) (33554432,0.933642) (67108864,0.648818) (1.34218e+08,0.553996) (2.68435e+08,0.530592) (5.36871e+08,0.504838) (1.07374e+09,0.483794) (2.14748e+09,0.468353) (4.29497e+09,0.452822) (8.58993e+09,0.467158) (1.71799e+10,0.459397) };
      \addlegendentry{\compiparassrsort};

      \legend{}
      
      \nextgroupplot[
      title=\distzipf \ulong,
      legend to name=legendplotp128,
      legend style={at={($(0,0)+(1cm,1cm)$)},legend columns=5,fill=none,draw=black,anchor=center,align=center},
      ]
      %% MULTIPLOT(algo|ptitle) select title as ptitle, size as x, 1000000.0 * milli * threads / (dsize * size * log(2, size)) as y, MULTIPLOT
      %% from pavgnames
      %% where algo not like 'mcstlmwm' and algo not like 'tbbparallelsort' and  algo not like 'ps4oparallel' and machine like 'i10pc136' and datatype like 'uint64' and size >= 2^13
      %% and (algo not like 'tbbparallelsort' or 1000000.0 * milli * threads / (dsize * size * log(2, size)) <= 2)
      %% and gen like 'zipf'
      %% order by titleorder, x
      \addplot coordinates { (8192,36.1966) (16384,32.4194) (32768,27.1436) (65536,27.6662) (131072,86.8874) (262144,45.034) (524288,25.6556) (1048576,11.53) (2097152,6.64956) (4194304,3.48543) (8388608,1.77403) (16777216,1.07715) (33554432,0.771795) (67108864,0.526902) (1.34218e+08,0.458399) (2.68435e+08,0.412474) (5.36871e+08,0.387024) (1.07374e+09,0.369266) (2.14748e+09,0.351425) (4.29497e+09,0.337114) (8.58993e+09,0.337532) (1.71799e+10,0.333214) };
      \addlegendentry{\compiparassssort};
      \addplot coordinates { (8192,54.1602) (16384,40.58) (32768,18.082) (65536,9.7281) (131072,6.65717) (262144,4.42806) (524288,3.1296) (1048576,2.40925) (2097152,2.19648) (4194304,1.54712) (8388608,1.22663) (16777216,1.02486) (33554432,0.920882) (67108864,0.780842) (1.34218e+08,0.726278) (2.68435e+08,0.669518) (5.36871e+08,0.649142) (1.07374e+09,0.6264) (2.14748e+09,0.627707) (4.29497e+09,0.628573) (8.58993e+09,0.632859) (1.71799e+10,0.637745) };
      \addlegendentry{\compppbbs};
      \addplot coordinates { (8192,2384.86) (16384,1042.25) (32768,552.623) (65536,260.111) (131072,109.628) (262144,438.117) (524288,234.449) (1048576,104.99) (2097152,48.6149) (4194304,22.2979) (8388608,10.4817) (16777216,4.97589) (33554432,2.89091) (67108864,1.7973) (1.34218e+08,1.33934) (2.68435e+08,1.17285) (5.36871e+08,1.15788) (1.07374e+09,1.23579) (2.14748e+09,1.32224) (4.29497e+09,1.27471) (8.58993e+09,1.33615) (1.71799e+10,1.29088) };
      \addlegendentry{\comppbalancedsort};
      \addplot coordinates { (8192,103.249) (16384,90.7624) (32768,60.0491) (65536,54.8689) (131072,34.3412) (262144,30.0183) (524288,24.4683) (1048576,18.9914) (2097152,17.2058) (4194304,13.2135) (8388608,11.4263) (16777216,10.4563) (33554432,9.82631) (67108864,8.89111) (1.34218e+08,8.19256) (2.68435e+08,8.38057) (5.36871e+08,8.06528) (1.07374e+09,7.60322) (2.14748e+09,7.22283) (4.29497e+09,7.7656) (8.58993e+09,7.44719) (1.71799e+10,7.74187) };
      \addlegendentry{\radixppbbr};
      \addplot coordinates { (8192,63200.9) (16384,47341.2) (32768,29776) (65536,13942.4) (131072,6600.71) (262144,3109.98) (524288,1471.21) (1048576,697.817) (2097152,335.323) (4194304,162.048) (8388608,79.1407) (16777216,39.6741) (33554432,19.3857) (67108864,9.85903) (1.34218e+08,5.15125) (2.68435e+08,2.88765) (5.36871e+08,1.78464) (1.07374e+09,1.25153) (2.14748e+09,1.00413) (4.29497e+09,0.888089) (8.58993e+09,0.81296) (1.71799e+10,0.766763) };
      \addlegendentry{\radixraduls};
      \addplot coordinates { (8192,218.682) (16384,183.365) (32768,195.897) (65536,236.915) (131072,139.174) (262144,75.3958) (524288,38.917) (1048576,19.8691) (2097152,10.084) (4194304,4.89236) (8388608,2.52124) (16777216,1.6015) (33554432,1.18196) (67108864,0.878887) (1.34218e+08,0.690433) (2.68435e+08,0.576255) (5.36871e+08,0.506001) (1.07374e+09,0.460937) (2.14748e+09,0.424409) (4.29497e+09,0.400972) (8.58993e+09,0.471235) (1.71799e+10,0.51801) };
      \addlegendentry{\radixregion};
      \addplot coordinates { (8192,29.5549) (16384,25.6764) (32768,21.6772) (65536,19.6878) (131072,159.757) (262144,82.9385) (524288,40.3762) (1048576,14.9867) (2097152,9.46194) (4194304,4.04622) (8388608,2.15672) (16777216,1.60647) (33554432,1.10481) (67108864,0.675729) (1.34218e+08,0.52122) (2.68435e+08,0.476691) (5.36871e+08,0.449004) (1.07374e+09,0.452472) (2.14748e+09,0.432446) (4.29497e+09,0.426968) (8.58993e+09,0.413223) (1.71799e+10,0.401608) };
      \addlegendentry{\compiparassrsort};

      \nextgroupplot[
      xlabel={Item count $n$},
      title=\distuniform \pair,
      xmin=2^16,
      xmax=2^33,
      xtick={2^16, 2^20, 2^24, 2^28, 2^32},
      ]
      %% MULTIPLOT(algo|ptitle) select title as ptitle, size as x, 1000000.0 * milli * threads / (dsize * size * log(2, size)) as y, MULTIPLOT
      %% from pavgnames
      %% where algo not like 'mcstlmwm' and algo not like 'tbbparallelsort' and  algo not like 'ps4oparallel' and machine like 'i10pc136' and datatype like 'pair' and size >= 2^13
      %% and gen like 'random'
      %% and (algo not like 'tbbparallelsort' or 1000000.0 * milli * threads / (dsize * size * log(2, size)) <= 2)
      %% order by titleorder, x
      \addplot coordinates { (8192,24.1445) (16384,20.2444) (32768,17.9417) (65536,113.886) (131072,43.9269) (262144,23.9562) (524288,11.8204) (1048576,6.23922) (2097152,3.46318) (4194304,2.03213) (8388608,1.25263) (16777216,0.803502) (33554432,0.554994) (67108864,0.501226) (1.34218e+08,0.474283) (2.68435e+08,0.44393) (5.36871e+08,0.431902) (1.07374e+09,0.406335) (2.14748e+09,0.401563) (4.29497e+09,0.394055) (8.58993e+09,0.40359) };
      \addlegendentry{\compiparassssort};
      \addplot coordinates { (8192,32.6756) (16384,43.7308) (32768,11.3342) (65536,6.8214) (131072,4.01981) (262144,2.73279) (524288,3.27673) (1048576,1.58723) (2097152,1.61207) (4194304,1.46805) (8388608,1.13841) (16777216,0.922579) (33554432,0.761547) (67108864,0.710915) (1.34218e+08,0.633581) (2.68435e+08,0.634246) (5.36871e+08,0.590317) (1.07374e+09,0.604756) (2.14748e+09,0.573471) (4.29497e+09,0.591547) (8.58993e+09,0.569484) };
      \addlegendentry{\compppbbs};
      \addplot coordinates { (8192,995.41) (16384,494.174) (32768,287.133) (65536,135.208) (131072,68.9655) (262144,213.741) (524288,110.538) (1048576,51.0478) (2097152,24.7297) (4194304,12.1082) (8388608,6.06479) (16777216,3.32227) (33554432,1.76041) (67108864,1.15061) (1.34218e+08,1.06002) (2.68435e+08,0.997722) (5.36871e+08,1.07607) (1.07374e+09,1.04045) (2.14748e+09,1.08949) (4.29497e+09,1.12244) (8.58993e+09,1.15045) };
      \addlegendentry{\comppbalancedsort};
      \addplot coordinates { (8192,49.002) (16384,15.1661) (32768,8.35517) (65536,4.79717) (131072,2.90643) (262144,2.17396) (524288,1.66361) (1048576,1.4801) (2097152,1.18425) (4194304,1.11718) (8388608,0.831269) (16777216,0.510637) (33554432,0.474466) (67108864,0.46266) (1.34218e+08,0.450479) (2.68435e+08,0.450053) (5.36871e+08,0.449507) (1.07374e+09,0.428418) (2.14748e+09,0.429907) (4.29497e+09,0.450367) (8.58993e+09,0.459117) };
      \addlegendentry{\radixppbbr};
      \addplot coordinates { (8192,26422.2) (16384,12234.8) (32768,5752.38) (65536,2684.32) (131072,1268.89) (262144,598.122) (524288,280.377) (1048576,133.39) (2097152,65.7323) (4194304,32.5722) (8388608,16.949) (16777216,7.83498) (33554432,4.2604) (67108864,2.70411) (1.34218e+08,1.9265) (2.68435e+08,1.56721) (5.36871e+08,1.35733) (1.07374e+09,1.21336) (2.14748e+09,1.1381) (4.29497e+09,1.10792) (8.58993e+09,1.09737) };
      \addlegendentry{\radixraduls};
      \addplot coordinates { (8192,98.6901) (16384,101.277) (32768,306.066) (65536,169.064) (131072,92.6343) (262144,47.4213) (524288,24.2547) (1048576,12.2773) (2097152,6.42282) (4194304,3.38013) (8388608,2.1632) (16777216,1.59197) (33554432,0.993033) (67108864,0.803281) (1.34218e+08,0.692315) (2.68435e+08,0.60825) (5.36871e+08,0.54009) (1.07374e+09,0.487221) (2.14748e+09,0.400826) (4.29497e+09,0.418398) (8.58993e+09,0.47163) };
      \addlegendentry{\radixregion};
      \addplot coordinates { (8192,16.589) (16384,16.1751) (32768,17.3061) (65536,125.468) (131072,54.657) (262144,23.6972) (524288,11.3562) (1048576,6.49197) (2097152,3.53419) (4194304,1.85651) (8388608,1.38064) (16777216,0.735718) (33554432,0.53258) (67108864,0.507633) (1.34218e+08,0.467831) (2.68435e+08,0.431448) (5.36871e+08,0.415812) (1.07374e+09,0.408319) (2.14748e+09,0.40866) (4.29497e+09,0.393821) (8.58993e+09,0.402385) };
      \addlegendentry{\compiparassrsort};

      \legend{}
      
      \nextgroupplot[
      xlabel={Item count $n$},
      title=\distuniform \bytes,
      every legend/.append style={at=(ticklabel cs:1.1)},
      xmax=2^31.5,
      xmin=2^14,
      xmax=2^30,
      xtick={2^14, 2^18, 2^22, 2^26, 2^30},
      ]
      %% MULTIPLOT(algo|ptitle) select title as ptitle, size as x, 1000000.0 * milli * threads / (dsize * size * log(2, size)) as y, MULTIPLOT
      %% from pavgnames
      %% where algo not like 'mcstlmwm' and algo not like 'tbbparallelsort' and  algo not like 'ps4oparallel' and machine like 'i10pc136' and datatype like 'byte' and size >= 2^13
      %% and gen like 'byte'
      %% and (algo not like 'tbbparallelsort' or 1000000.0 * milli * threads / (dsize * size * log(2, size)) <= 2)
      %% order by titleorder, x
      \addplot coordinates { (8192,14.2094) (16384,93.7758) (32768,51.1977) (65536,25.2332) (131072,8.41195) (262144,4.31916) (524288,2.48714) (1048576,1.43759) (2097152,0.93888) (4194304,0.595164) (8388608,0.532194) (16777216,0.519199) (33554432,0.485471) (67108864,0.44711) (1.34218e+08,0.430104) (2.68435e+08,0.398026) (5.36871e+08,0.394783) (1.07374e+09,0.402594) };
      \addlegendentry{\compiparassssort};
      \addplot coordinates { (8192,14.9815) (16384,7.27182) (32768,3.43113) (65536,2.45456) (131072,1.80535) (262144,1.51521) (524288,1.3583) (1048576,1.1653) (2097152,0.774754) (4194304,0.56685) (8388608,0.507996) (16777216,0.480067) (33554432,0.464908) (67108864,0.462459) (1.34218e+08,0.464291) (2.68435e+08,0.474112) (5.36871e+08,0.489275) (1.07374e+09,0.504764) };
      \addlegendentry{\compppbbs};
      \addplot coordinates { (8192,201.802) (16384,86.9239) (32768,41.536) (65536,25.6209) (131072,13.3809) (262144,31.3147) (524288,17.6572) (1048576,8.97279) (2097152,4.09743) (4194304,2.25949) (8388608,1.40547) (16777216,1.07803) (33554432,0.967755) (67108864,0.935842) (1.34218e+08,0.929917) (2.68435e+08,0.949882) (5.36871e+08,0.976143) (1.07374e+09,0.992358) };
      \addlegendentry{\comppbalancedsort};

      \legend{}
      
    \end{groupplot}
    \coordinate (c3) at ($(c1)!.5!(c2)$);
    \node[below] at (c3 |- current bounding box.south)
    {\pgfplotslegendfromname{legendplotp128}};
  \end{tikzpicture}

  \caption{
    Running times of parallel algorithms on different input distributions and data types of size $D$ executed on machine \pcintelfour.
    A horizontal (vertical) line in the bottom left corner indicates that the algorithm's running time was too large for the plot (that the algorithm's interface does not accept the data type).
  }
  \label{fig:par rt distr types 128}
\end{figure}

\begin{figure}[tbp]
%% drop view if exists p CASCADE;
%% create view p as
%% select * from pradixalgoswithips4oml
%% union
%% select * from pcomparisonalgos

%% SQL
%% drop view if exists palgos1 CASCADE;
%% create view palgos1 as
%% select avgparallel.* from avgparallel inner join p
%% on avgparallel.algo = p.algo

%% SQL
%% drop view if exists palgos2 CASCADE;
%% create view palgos2 as
%% select * from palgos1 natural join pfast

%% SQL
%% drop view if exists palgos CASCADE;
%% create view palgos as
%% select machine, gen, datatype, algo, parallel, threads, vector, size, meminterleaved, copyback, milli
%% from palgos2

%% SQL
%% drop view if exists pavg CASCADE;
%% create view pavg as
%% select machine, algo, size, datatype, gen, threads, AVG(milli) as milli
%% from palgos
%% group by machine, algo, size, datatype, gen, threads


%% SQL
%% drop table if exists undisplayed CASCADE;
%% create table undisplayed(
%% machine character varying,
%% algo character varying,
%% size            bigint,
%% datatype character varying,
%% gen character varying,
%% threads            bigint,
%% milli  double precision
%% )

%% SQL
%% insert into undisplayed (machine, algo, datatype, gen, threads, size, milli)
%% values
%% ('i10pc132', 'pbbsradixsort', 'uint64', 'almostsorted', 1, -1, 0.1*2^19.5*19.5*8/1000000),
%% ('i10pc132', 'pbbsradixsort', 'uint64', 'almostsorted', 1, -2, 0.1*2^20.5*20.5*8/1000000),
%% ('i10pc132', 'pbbsradixsort', 'uint64', 'almostsorted', 1, -3, 0.1*2^21.5*21.5*8/1000000),
%% ('i10pc132', 'pbbsradixsort', 'uint64', 'rootdupls', 1, -1, 0.1*2^19.5*19.5*8/1000000),
%% ('i10pc132', 'pbbsradixsort', 'uint64', 'rootdupls', 1, -2, 0.1*2^20.5*20.5*8/1000000),
%% ('i10pc132', 'pbbsradixsort', 'uint64', 'rootdupls', 1, -3, 0.1*2^21.5*21.5*8/1000000),
%% ('i10pc132', 'pbbsradixsort', 'uint64', 'twicedupes', 1, -1, 0.3*2^19.5*19.5*8/1000000),
%% ('i10pc132', 'pbbsradixsort', 'uint64', 'twicedupes', 1, -2, 0.3*2^20.5*20.5*8/1000000),
%% ('i10pc132', 'pbbsradixsort', 'uint64', 'twicedupes', 1, -3, 0.3*2^21.5*21.5*8/1000000),
%% ('i10pc132', 'pbbsradixsort', 'uint64', 'zipf', 1, -1, 0.3*2^19.5*19.5*8/1000000),
%% ('i10pc132', 'pbbsradixsort', 'uint64', 'zipf', 1, -2, 0.3*2^20.5*20.5*8/1000000),
%% ('i10pc132', 'pbbsradixsort', 'uint64', 'zipf', 1, -3, 0.3*2^21.5*21.5*8/1000000),
%% ('i10pc132', 'pbbsradixsort', 'double', 'random', 1, -1, 0.5*2^19*19*8/1000000),
%% ('i10pc132', 'pbbsradixsort', 'double', 'random', 1, -2, 0.3*2^19*19*8/1000000),
%% ('i10pc132', 'pbbsradixsort', 'double', 'random', 1, -3, 0.1*2^19*19*8/1000000),
%% ('i10pc132', 'raduls', 'double', 'random', 1, -1, 0.5*2^19.5*19.5*8/1000000),
%% ('i10pc132', 'raduls', 'double', 'random', 1, -2, 0.3*2^19.5*19.5*8/1000000),
%% ('i10pc132', 'raduls', 'double', 'random', 1, -3, 0.1*2^19.5*19.5*8/1000000),
%% ('i10pc132', 'regionsort', 'double', 'random', 1, -1, 0.5*2^20*20*8/1000000),
%% ('i10pc132', 'regionsort', 'double', 'random', 1, -2, 0.3*2^20*20*8/1000000),
%% ('i10pc132', 'regionsort', 'double', 'random', 1, -3, 0.1*2^20*20*8/1000000),
%% ('i10pc132', 'pbbsradixsort', 'byte', 'byte', 1, -1, 0.5*2^15.5*15.5*100/1000000),
%% ('i10pc132', 'pbbsradixsort', 'byte', 'byte', 1, -2, 0.3*2^15.5*15.5*100/1000000),
%% ('i10pc132', 'pbbsradixsort', 'byte', 'byte', 1, -3, 0.1*2^15.5*15.5*100/1000000),
%% ('i10pc132', 'raduls', 'byte', 'byte', 1, -1, 0.5*2^16*16*100/1000000),
%% ('i10pc132', 'raduls', 'byte', 'byte', 1, -2, 0.3*2^16*16*100/1000000),
%% ('i10pc132', 'raduls', 'byte', 'byte', 1, -3, 0.1*2^16*16*100/1000000),
%% ('i10pc132', 'regionsort', 'byte', 'byte', 1, -1, 0.5*2^16.5*16.5*100/1000000),
%% ('i10pc132', 'regionsort', 'byte', 'byte', 1, -2, 0.3*2^16.5*16.5*100/1000000),
%% ('i10pc132', 'regionsort', 'byte', 'byte', 1, -3, 0.1*2^16.5*16.5*100/1000000)

%% SQL
%% drop view if exists pavg1 CASCADE;
%% create view pavg1 as
%% select * from undisplayed union select * from pavg

%% SQL
%% drop view if exists pavgnames CASCADE;
%% create view pavgnames as
%% select pavg1.*, titles.title, titles.titleorder, datatypesizes.dsize from pavg1
%% inner join titles
%% on titles.algo like pavg1.algo
%% inner join datatypesizes
%% on pavg1.datatype = datatypesizes.datatype

  \begin{tikzpicture}
    \begin{groupplot}[
      group style={
        group size=2 by 4,
        vertical sep=1.2cm,
      },
      width=0.5\textwidth,
      height=0.22\textheight,
      xmode=log,
      log base x=2,
      plotstyleparallel,
      xmajorgrids=true,
      ymajorgrids=true,
      ymin=0,
      ymax=1.5,
      xmin=2^18,
      xmax=2^34,
      xtick={2^18, 2^22, 2^26, 2^30, 2^34},
      restrict y to domain=-1:30,
      every axis title/.append style={yshift=-0.2cm},
      ]
      
      
      \nextgroupplot[
      title=\distuniform \double]
      %% MULTIPLOT(algo|ptitle) select title as ptitle, size as x, 1000000.0 * milli * threads / (dsize * size * log(2, size)) as y, MULTIPLOT
      %% from pavgnames
      %% where algo not like 'aspasparallel' and algo not like 'mcstlmwm' and algo not like 'tbbparallelsort' and  algo not like 'ps4oparallel' and machine like 'i10pc132' and datatype like 'double' and size >= 2^13
      %% and (algo not like 'tbbparallelsort' or 1000000.0 * milli * threads / (dsize * size * log(2, size)) <= 2.5)
      %% and gen like 'random'
      %% order by titleorder, x    
      \addplot coordinates { (8192,27.5857) (16384,25.1965) (32768,23.5532) (65536,21.4295) (131072,12.8192) (262144,6.18456) (524288,5.47429) (1048576,2.87095) (2097152,1.69749) (4194304,1.13218) (8388608,0.91102) (16777216,0.738083) (33554432,0.654619) (67108864,0.589737) (1.34218e+08,0.555911) (2.68435e+08,0.534387) (5.36871e+08,0.516994) (1.07374e+09,0.512044) (2.14748e+09,0.49188) (4.29497e+09,0.487647) (8.58993e+09,0.483887) (1.71799e+10,0.48235) };
      \addlegendentry{\compiparassssort};
      \addplot coordinates { (8192,39.7038) (16384,60.172) (32768,12.7315) (65536,7.51415) (131072,5.58886) (262144,3.49268) (524288,2.44167) (1048576,1.92412) (2097152,1.50526) (4194304,1.38539) (8388608,1.37186) (16777216,1.30086) (33554432,1.27455) (67108864,1.23773) (1.34218e+08,1.23116) (2.68435e+08,1.20887) (5.36871e+08,1.23598) (1.07374e+09,1.20994) (2.14748e+09,1.19039) (4.29497e+09,1.17536) (8.58993e+09,1.16639) (1.71799e+10,1.15632) };
      \addlegendentry{\compppbbs};
      \addplot coordinates { (8192,847.15) (16384,438.591) (32768,208.042) (65536,79.5182) (131072,305.062) (262144,160.619) (524288,71.2077) (1048576,38.5281) (2097152,15.0501) (4194304,7.8912) (8388608,4.01555) (16777216,2.17749) (33554432,1.73251) (67108864,1.51162) (1.34218e+08,1.39952) (2.68435e+08,1.35812) (5.36871e+08,1.33518) (1.07374e+09,1.32896) (2.14748e+09,1.27715) (4.29497e+09,1.32577) (8.58993e+09,1.26516) (1.71799e+10,1.37106) };
      \addlegendentry{\comppbalancedsort};

      \legend{}
      
      % (Relative) Coordinate at top of the first plot
      \coordinate (c1) at (rel axis cs:0,1);
      
      \nextgroupplot[
      title=\distalmostsorted \ulong]
      %% MULTIPLOT(algo|ptitle) select title as ptitle, size as x, 1000000.0 * milli * threads / (dsize * size * log(2, size)) as y, MULTIPLOT
      %% from pavgnames
      %% where algo not like 'mcstlmwm' and algo not like 'tbbparallelsort' and  algo not like 'ps4oparallel' and machine like 'i10pc132' and datatype like 'uint64' and size >= 2^13
      %% and gen like 'almostsorted'
      %% order by titleorder, x
      \addplot coordinates { (8192,16.0499) (16384,15.1167) (32768,14.3268) (65536,31.8157) (131072,9.04455) (262144,5.3755) (524288,3.55448) (1048576,1.78659) (2097152,0.9832) (4194304,0.675057) (8388608,0.568188) (16777216,0.47338) (33554432,0.419561) (67108864,0.367063) (1.34218e+08,0.351318) (2.68435e+08,0.335594) (5.36871e+08,0.321604) (1.07374e+09,0.312914) (2.14748e+09,0.309367) (4.29497e+09,0.307142) (8.58993e+09,0.30859) (1.71799e+10,0.308409) };
      \addlegendentry{\compiparassssort};
      \addplot coordinates { (8192,20.5829) (16384,25.1469) (32768,8.83981) (65536,4.88491) (131072,2.87548) (262144,1.90965) (524288,1.3044) (1048576,0.95974) (2097152,0.759786) (4194304,0.679352) (8388608,0.640863) (16777216,0.576664) (33554432,0.541347) (67108864,0.508715) (1.34218e+08,0.497402) (2.68435e+08,0.48414) (5.36871e+08,0.492089) (1.07374e+09,0.466749) (2.14748e+09,0.459775) (4.29497e+09,0.450267) (8.58993e+09,0.444935) (1.71799e+10,0.435903) };
      \addlegendentry{\compppbbs};
      \addplot coordinates { (8192,910.079) (16384,498.247) (32768,188.389) (65536,83.5246) (131072,260.135) (262144,119.322) (524288,50.7119) (1048576,27.8935) (2097152,12.6163) (4194304,5.11445) (8388608,2.67068) (16777216,1.35835) (33554432,0.811368) (67108864,0.538272) (1.34218e+08,0.488948) (2.68435e+08,0.46328) (5.36871e+08,0.46074) (1.07374e+09,0.457607) (2.14748e+09,0.467484) (4.29497e+09,0.494448) (8.58993e+09,0.484282) (1.71799e+10,0.503622) };
      \addlegendentry{\comppbalancedsort};
      \addplot coordinates { (8192,132.386) (16384,39.0923) (32768,20.6861) (65536,12.0568) (131072,7.43071) (262144,5.09353) (524288,3.52912) (1048576,2.93002) (2097152,2.55953) (4194304,2.48478) (8388608,2.8091) (16777216,3.21684) (33554432,3.11804) (67108864,2.64176) (1.34218e+08,2.38707) (2.68435e+08,2.79577) (5.36871e+08,2.71605) (1.07374e+09,2.97276) (2.14748e+09,2.43485) (4.29497e+09,3.02488) (8.58993e+09,2.91521) };
      \addlegendentry{\radixppbbr};
      \addplot coordinates { (8192,21022.1) (16384,9721.16) (32768,4554.28) (65536,2132.03) (131072,1124) (262144,645.354) (524288,411.379) (1048576,298.41) (2097152,238.877) (4194304,21.4998) (8388608,10.5885) (16777216,5.50656) (33554432,3.45635) (67108864,2.43904) (1.34218e+08,1.93743) (2.68435e+08,1.6674) (5.36871e+08,1.50838) (1.07374e+09,0.859171) (2.14748e+09,0.787538) (4.29497e+09,0.762352) (8.58993e+09,0.732161) (1.71799e+10,0.703249) };
      \addlegendentry{\radixraduls};
      \addplot coordinates { (8192,64.2218) (16384,37.0942) (32768,47.248) (65536,35.8742) (131072,8.19564) (262144,4.33301) (524288,2.66305) (1048576,1.47087) (2097152,0.892049) (4194304,0.574651) (8388608,0.588492) (16777216,0.459595) (33554432,0.36621) (67108864,0.29292) (1.34218e+08,0.398758) (2.68435e+08,0.305654) (5.36871e+08,0.26038) (1.07374e+09,0.235056) (2.14748e+09,0.220893) (4.29497e+09,0.218339) (8.58993e+09,0.273865) (1.71799e+10,0.258642) };
      \addlegendentry{\radixregion};
      \addplot coordinates { (8192,15.3207) (16384,13.1184) (32768,11.6114) (65536,10.7624) (131072,8.94331) (262144,6.23969) (524288,3.988) (1048576,1.91019) (2097152,1.32201) (4194304,0.738625) (8388608,0.472116) (16777216,0.345004) (33554432,0.503242) (67108864,0.457855) (1.34218e+08,0.402935) (2.68435e+08,0.346924) (5.36871e+08,0.309447) (1.07374e+09,0.282464) (2.14748e+09,0.251848) (4.29497e+09,0.23369) (8.58993e+09,0.390851) (1.71799e+10,0.4591) };
      \addlegendentry{\compiparassrsort};

      \legend{}
      
      % (Relative) Coordinate at top of the second plot
      \coordinate (c2) at (rel axis cs:1,1);
      
      \nextgroupplot[
      y unit=ns,
      title=\distduplicatesroot \ulong]
      %% MULTIPLOT(algo|ptitle) select title as ptitle, size as x, 1000000.0 * milli * threads / (dsize * size * log(2, size)) as y, MULTIPLOT
      %% from pavgnames
      %% where algo not like 'mcstlmwm' and algo not like 'tbbparallelsort' and  algo not like 'ps4oparallel' and machine like 'i10pc132' and datatype like 'uint64' and size >= 2^13
      %% and gen like 'rootdupls'
      %% order by titleorder, x
      \addplot coordinates { (8192,12.2426) (16384,11.4228) (32768,10.1894) (65536,25.9571) (131072,15.1346) (262144,6.37949) (524288,8.76884) (1048576,2.78727) (2097152,1.48339) (4194304,0.861083) (8388608,0.57843) (16777216,0.468663) (33554432,0.390337) (67108864,0.34555) (1.34218e+08,0.312998) (2.68435e+08,0.299531) (5.36871e+08,0.296582) (1.07374e+09,0.304806) (2.14748e+09,0.302222) (4.29497e+09,0.313835) (8.58993e+09,0.314906) (1.71799e+10,0.327205) };
      \addlegendentry{\compiparassssort};
      \addplot coordinates { (8192,25.046) (16384,78.9204) (32768,12.6855) (65536,4.96645) (131072,3.26625) (262144,1.70739) (524288,1.23748) (1048576,1.04423) (2097152,0.765885) (4194304,0.632304) (8388608,0.627833) (16777216,0.569667) (33554432,0.554926) (67108864,0.514538) (1.34218e+08,0.511242) (2.68435e+08,0.489322) (5.36871e+08,0.513405) (1.07374e+09,0.490934) (2.14748e+09,0.50603) (4.29497e+09,0.49495) (8.58993e+09,0.49565) (1.71799e+10,0.480736) };
      \addlegendentry{\compppbbs};
      \addplot coordinates { (8192,1234.79) (16384,481.806) (32768,204.631) (65536,92.2784) (131072,835.796) (262144,248.831) (524288,116.187) (1048576,63.656) (2097152,25.5673) (4194304,13.3966) (8388608,5.90824) (16777216,3.64796) (33554432,1.30564) (67108864,1.43349) (1.34218e+08,0.940195) (2.68435e+08,1.09606) (5.36871e+08,0.835837) (1.07374e+09,1.00511) (2.14748e+09,0.888722) (4.29497e+09,1.18484) (8.58993e+09,0.820021) (1.71799e+10,0.948558) };
      \addlegendentry{\comppbalancedsort};
      \addplot coordinates { (8192,140.282) (16384,60.6927) (32768,20.5069) (65536,13.6904) (131072,8.51332) (262144,6.14339) (524288,4.44041) (1048576,3.38868) (2097152,3.17215) (4194304,2.7876) (8388608,3.39381) (16777216,3.35566) (33554432,3.45227) (67108864,2.87588) (1.34218e+08,3.45287) (2.68435e+08,3.06927) (5.36871e+08,3.01066) (1.07374e+09,2.50702) (2.14748e+09,2.88543) (4.29497e+09,3.86111) (8.58993e+09,3.62493) };
      \addlegendentry{\radixppbbr};
      \addplot coordinates { (8192,24339.9) (16384,11220.1) (32768,5178.9) (65536,2426.04) (131072,1250.75) (262144,592.085) (524288,309.153) (1048576,159.949) (2097152,87.4777) (4194304,48.5113) (8388608,28.9602) (16777216,17.0825) (33554432,10.8473) (67108864,7.21133) (1.34218e+08,4.94101) (2.68435e+08,1.22731) (5.36871e+08,1.04215) (1.07374e+09,0.94747) (2.14748e+09,0.875392) (4.29497e+09,0.836601) (8.58993e+09,0.803613) (1.71799e+10,0.78771) };
      \addlegendentry{\radixraduls};
      \addplot coordinates { (8192,41.5232) (16384,22.9464) (32768,243.356) (65536,22.3893) (131072,17.8682) (262144,5.43349) (524288,4.41182) (1048576,1.66543) (2097152,1.28338) (4194304,0.650535) (8388608,0.532682) (16777216,0.447374) (33554432,0.40282) (67108864,0.414562) (1.34218e+08,0.380117) (2.68435e+08,0.350761) (5.36871e+08,0.339826) (1.07374e+09,0.323112) (2.14748e+09,0.314947) (4.29497e+09,0.395651) (8.58993e+09,0.43738) (1.71799e+10,0.469224) };
      \addlegendentry{\radixregion};
      \addplot coordinates { (8192,17.1364) (16384,16.8372) (32768,16.7431) (65536,66.7431) (131072,26.2486) (262144,12.6762) (524288,5.90382) (1048576,3.17982) (2097152,1.68148) (4194304,1.0762) (8388608,0.756489) (16777216,0.575844) (33554432,0.491252) (67108864,0.395656) (1.34218e+08,0.390257) (2.68435e+08,0.342066) (5.36871e+08,0.361705) (1.07374e+09,0.324391) (2.14748e+09,0.319403) (4.29497e+09,0.517051) (8.58993e+09,0.467913) (1.71799e+10,0.479041) };
      \addlegendentry{\compiparassrsort};

      \legend{}

      \nextgroupplot[
      title=\distduplicatestwice \ulong,
      ]
      %% MULTIPLOT(algo|ptitle) select title as ptitle, size as x, 1000000.0 * milli * threads / (dsize * size * log(2, size)) as y, MULTIPLOT
      %% from pavgnames
      %% where algo not like 'mcstlmwm' and algo not like 'tbbparallelsort' and  algo not like 'ps4oparallel' and machine like 'i10pc132' and datatype like 'uint64' and size >= 2^13
      %% and (algo not like 'tbbparallelsort' or 1000000.0 * milli * threads / (dsize * size * log(2, size)) <= 2)
      %% and gen like 'twicedupes'
      %% order by titleorder, x
      \addplot coordinates { (8192,24.8765) (16384,22.0782) (32768,20.8616) (65536,37.8762) (131072,13.8647) (262144,8.34454) (524288,5.26204) (1048576,2.82479) (2097152,1.64511) (4194304,1.1377) (8388608,0.892436) (16777216,0.709457) (33554432,0.626659) (67108864,0.555338) (1.34218e+08,0.529728) (2.68435e+08,0.509298) (5.36871e+08,0.489358) (1.07374e+09,0.483349) (2.14748e+09,0.466281) (4.29497e+09,0.462866) (8.58993e+09,0.461789) (1.71799e+10,0.460529) };
      \addlegendentry{\compiparassssort};
      \addplot coordinates { (8192,97.678) (16384,40.3356) (32768,10.5454) (65536,6.24357) (131072,3.75049) (262144,2.82519) (524288,2.108) (1048576,1.66062) (2097152,1.25141) (4194304,1.28742) (8388608,1.16497) (16777216,1.09934) (33554432,1.08128) (67108864,1.04281) (1.34218e+08,1.03729) (2.68435e+08,1.02142) (5.36871e+08,1.0425) (1.07374e+09,1.01324) (2.14748e+09,1.01532) (4.29497e+09,0.999146) (8.58993e+09,0.990741) (1.71799e+10,0.985001) };
      \addlegendentry{\compppbbs};
      \addplot coordinates { (8192,886.394) (16384,432.092) (32768,217.315) (65536,94.4707) (131072,348.906) (262144,148.528) (524288,60.5384) (1048576,33.9161) (2097152,13.2868) (4194304,7.39575) (8388608,4.013) (16777216,2.2336) (33554432,1.61011) (67108864,1.41) (1.34218e+08,1.3189) (2.68435e+08,1.28959) (5.36871e+08,1.24225) (1.07374e+09,1.23205) (2.14748e+09,1.22602) (4.29497e+09,1.23156) (8.58993e+09,1.20698) (1.71799e+10,1.23938) };
      \addlegendentry{\comppbalancedsort};
      \addplot coordinates { (8192,55.1248) (16384,37.7392) (32768,25.5499) (65536,17.8561) (131072,11.9525) (262144,9.78468) (524288,8.29842) (1048576,6.468) (2097152,5.56912) (4194304,6.00674) (8388608,6.09337) (16777216,6.27409) (33554432,5.41566) (67108864,4.33926) (1.34218e+08,4.06446) (2.68435e+08,6.01851) (5.36871e+08,5.07067) (1.07374e+09,7.30813) (2.14748e+09,4.55513) (4.29497e+09,3.63823) (8.58993e+09,2.79339) };
      \addlegendentry{\radixppbbr};
      \addplot coordinates { (8192,21396.2) (16384,9928.46) (32768,4639.82) (65536,2171.71) (131072,1146.37) (262144,657.178) (524288,413.585) (1048576,298.46) (2097152,239.477) (4194304,56.8105) (8388608,10.5879) (16777216,5.53244) (33554432,3.48854) (67108864,2.46642) (1.34218e+08,1.96081) (2.68435e+08,1.68707) (5.36871e+08,1.52972) (1.07374e+09,0.959876) (2.14748e+09,0.792664) (4.29497e+09,0.76248) (8.58993e+09,0.736411) (1.71799e+10,0.710103) };
      \addlegendentry{\radixraduls};
      \addplot coordinates { (8192,66.6842) (16384,38.9212) (32768,217.113) (65536,226.146) (131072,24.7144) (262144,12.2751) (524288,6.33168) (1048576,3.31027) (2097152,1.81129) (4194304,1.08326) (8388608,1.11724) (16777216,1.32941) (33554432,0.564218) (67108864,0.493243) (1.34218e+08,0.458223) (2.68435e+08,0.423818) (5.36871e+08,0.376396) (1.07374e+09,0.348821) (2.14748e+09,0.342006) (4.29497e+09,0.360947) (8.58993e+09,0.484257) (1.71799e+10,0.501439) };
      \addlegendentry{\radixregion};
      \addplot coordinates { (8192,16.7125) (16384,16.7258) (32768,17.4499) (65536,46.1293) (131072,23.8187) (262144,11.0921) (524288,5.37834) (1048576,2.74844) (2097152,1.48643) (4194304,0.957079) (8388608,0.709054) (16777216,0.580222) (33554432,0.691895) (67108864,0.701787) (1.34218e+08,0.542645) (2.68435e+08,0.426983) (5.36871e+08,0.404308) (1.07374e+09,0.397975) (2.14748e+09,0.377951) (4.29497e+09,0.381829) (8.58993e+09,0.548357) (1.71799e+10,0.582678) };
      \addlegendentry{\compiparassrsort};

      \legend{}
      
      \nextgroupplot[
      every axis y label/.append style={at=(ticklabel cs:1.1)},
      ylabel={Running time $t/ Dn \log_2 n$},
      title=\distexpo \ulong]
      %% MULTIPLOT(algo|ptitle) select title as ptitle, size as x, 1000000.0 * milli * threads / (dsize * size * log(2, size)) as y, MULTIPLOT
      %% from pavgnames
      %% where algo not like 'mcstlmwm' and algo not like 'tbbparallelsort' and  algo not like 'ps4oparallel' and machine like 'i10pc132' and datatype like 'uint64' and size >= 2^13
      %% and (algo not like 'tbbparallelsort' or 1000000.0 * milli * threads / (dsize * size * log(2, size)) <= 2.5)
      %% and gen like 'exponential'
      %% order by titleorder, x
      \addplot coordinates { (8192,21.999) (16384,19.933) (32768,17.5827) (65536,29.2695) (131072,21.8813) (262144,10.5519) (524288,8.78714) (1048576,4.54442) (2097152,2.47188) (4194304,1.48445) (8388608,1.00266) (16777216,0.718522) (33554432,0.563876) (67108864,0.48204) (1.34218e+08,0.447569) (2.68435e+08,0.416699) (5.36871e+08,0.395987) (1.07374e+09,0.37504) (2.14748e+09,0.364832) (4.29497e+09,0.359584) (8.58993e+09,0.360315) (1.71799e+10,0.356038) };
      \addlegendentry{\compiparassssort};
      \addplot coordinates { (8192,33.06) (16384,44.0567) (32768,10.7133) (65536,6.12645) (131072,3.55823) (262144,2.56239) (524288,1.84691) (1048576,1.33386) (2097152,1.10993) (4194304,1.04313) (8388608,0.986729) (16777216,0.924046) (33554432,0.885112) (67108864,0.849676) (1.34218e+08,0.835519) (2.68435e+08,0.831446) (5.36871e+08,0.82262) (1.07374e+09,0.788867) (2.14748e+09,0.782041) (4.29497e+09,0.75947) (8.58993e+09,0.757295) (1.71799e+10,0.737944) };
      \addlegendentry{\compppbbs};
      \addplot coordinates { (8192,964.777) (16384,456.629) (32768,201.269) (65536,111.517) (131072,274.472) (262144,141.412) (524288,69.6269) (1048576,36.144) (2097152,16.6623) (4194304,7.82408) (8388608,4.08331) (16777216,2.07125) (33554432,1.55522) (67108864,1.33072) (1.34218e+08,1.22737) (2.68435e+08,1.1496) (5.36871e+08,1.1054) (1.07374e+09,1.09093) (2.14748e+09,1.08646) (4.29497e+09,1.09273) (8.58993e+09,1.07345) (1.71799e+10,1.11889) };
      \addlegendentry{\comppbalancedsort};
      \addplot coordinates { (8192,30.1076) (16384,16.8373) (32768,8.54403) (65536,5.56433) (131072,3.98184) (262144,2.58331) (524288,1.60007) (1048576,1.25592) (2097152,0.746363) (4194304,0.739535) (8388608,0.684027) (16777216,0.528162) (33554432,0.540028) (67108864,0.55457) (1.34218e+08,0.538581) (2.68435e+08,0.583033) (5.36871e+08,0.649669) (1.07374e+09,0.667858) (2.14748e+09,0.621725) (4.29497e+09,0.612828) (8.58993e+09,0.626741) (1.71799e+10,0.664751) };
      \addlegendentry{\radixppbbr};
      \addplot coordinates { (8192,8642.25) (16384,8708.16) (32768,5953.48) (65536,1921.57) (131072,795.5) (262144,267.885) (524288,109.018) (1048576,50.436) (2097152,18.3362) (4194304,7.87417) (8388608,4.17977) (16777216,2.32151) (33554432,1.38971) (67108864,0.991354) (1.34218e+08,0.765782) (2.68435e+08,0.644455) (5.36871e+08,0.574848) (1.07374e+09,0.559688) (2.14748e+09,0.609176) (4.29497e+09,0.707112) (8.58993e+09,0.6669) (1.71799e+10,0.638504) };
      \addlegendentry{\radixraduls};
      \addplot coordinates { (8192,234.727) (16384,239.222) (32768,378.135) (65536,218.766) (131072,115.503) (262144,58.0932) (524288,28.7403) (1048576,14.2174) (2097152,7.14086) (4194304,3.59086) (8388608,2.08988) (16777216,1.36501) (33554432,0.997476) (67108864,0.768024) (1.34218e+08,0.66438) (2.68435e+08,0.614584) (5.36871e+08,0.573067) (1.07374e+09,0.566959) (2.14748e+09,0.548032) (4.29497e+09,0.499744) (8.58993e+09,0.480181) (1.71799e+10,0.477185) };
      \addlegendentry{\radixregion};
      \addplot coordinates { (8192,41.1086) (16384,29.5332) (32768,24.8083) (65536,56.6942) (131072,24.86) (262144,11.9188) (524288,5.8921) (1048576,3.06656) (2097152,1.74045) (4194304,1.15289) (8388608,0.730269) (16777216,0.619494) (33554432,0.504625) (67108864,0.464116) (1.34218e+08,0.44361) (2.68435e+08,0.436947) (5.36871e+08,0.416674) (1.07374e+09,0.410813) (2.14748e+09,0.389569) (4.29497e+09,0.394764) (8.58993e+09,0.41756) (1.71799e+10,0.416434) };
      \addlegendentry{\compiparassrsort};

      \legend{}
      
      \nextgroupplot[
      title=\distzipf \ulong,
      legend to name=legendplotp132,
      legend style={at={($(0,0)+(1cm,1cm)$)},legend columns=5,fill=none,draw=black,anchor=center,align=center},
      ]
      %% MULTIPLOT(algo|ptitle) select title as ptitle, size as x, 1000000.0 * milli * threads / (dsize * size * log(2, size)) as y, MULTIPLOT
      %% from pavgnames
      %% where algo not like 'mcstlmwm' and algo not like 'tbbparallelsort' and  algo not like 'ps4oparallel' and machine like 'i10pc132' and datatype like 'uint64' and size >= 2^13
      %% and (algo not like 'tbbparallelsort' or 1000000.0 * milli * threads / (dsize * size * log(2, size)) <= 2.5)
      %% and gen like 'zipf'
      %% order by titleorder, x
      \addplot coordinates { (8192,26.4353) (16384,23.6345) (32768,22.0047) (65536,21.1719) (131072,12.7903) (262144,6.49243) (524288,5.62361) (1048576,2.88288) (2097152,1.71664) (4194304,1.17738) (8388608,0.882472) (16777216,0.6292) (33554432,0.507294) (67108864,0.440624) (1.34218e+08,0.391677) (2.68435e+08,0.362755) (5.36871e+08,0.342017) (1.07374e+09,0.345118) (2.14748e+09,0.316489) (4.29497e+09,0.307875) (8.58993e+09,0.307853) (1.71799e+10,0.308562) };
      \addlegendentry{\compiparassssort};
      \addplot coordinates { (8192,42.5483) (16384,22.0942) (32768,12.4044) (65536,7.00968) (131072,4.57137) (262144,3.13301) (524288,2.2579) (1048576,1.58421) (2097152,1.32367) (4194304,1.1622) (8388608,1.10829) (16777216,1.0095) (33554432,0.947414) (67108864,0.881464) (1.34218e+08,0.849721) (2.68435e+08,0.836039) (5.36871e+08,0.80488) (1.07374e+09,0.764472) (2.14748e+09,0.751177) (4.29497e+09,0.742899) (8.58993e+09,0.737094) (1.71799e+10,0.729719) };
      \addlegendentry{\compppbbs};
      \addplot coordinates { (8192,845.771) (16384,408.155) (32768,179.017) (65536,114.47) (131072,289.26) (262144,136.63) (524288,61.9125) (1048576,33.1212) (2097152,17.0865) (4194304,6.9163) (8388608,3.98411) (16777216,2.0593) (33554432,1.48332) (67108864,1.24847) (1.34218e+08,1.15772) (2.68435e+08,1.09361) (5.36871e+08,1.05555) (1.07374e+09,1.16979) (2.14748e+09,1.09947) (4.29497e+09,1.0846) (8.58993e+09,1.11607) (1.71799e+10,1.15393) };
      \addlegendentry{\comppbalancedsort};
      \addplot coordinates { (8192,51.2648) (16384,38.9144) (32768,28.2259) (65536,22.3137) (131072,16.4956) (262144,14.1968) (524288,11.8505) (1048576,11.1749) (2097152,10.0018) (4194304,10.1098) (8388608,8.68499) (16777216,8.45838) (33554432,8.37927) (67108864,7.41827) (1.34218e+08,6.93854) (2.68435e+08,7.06345) (5.36871e+08,6.65825) (1.07374e+09,6.70816) (2.14748e+09,6.17861) (4.29497e+09,6.90814) (8.58993e+09,6.0606) };
      \addlegendentry{\radixppbbr};
      \addplot coordinates { (8192,31881.4) (16384,22228.8) (32768,14014.6) (65536,6578.97) (131072,3061.23) (262144,1445.17) (524288,684.401) (1048576,324.727) (2097152,156.013) (4194304,75.4769) (8388608,36.9744) (16777216,18.7493) (33554432,9.33121) (67108864,5.02243) (1.34218e+08,2.89569) (2.68435e+08,1.85688) (5.36871e+08,1.35076) (1.07374e+09,1.09519) (2.14748e+09,0.963807) (4.29497e+09,0.890822) (8.58993e+09,0.830644) (1.71799e+10,0.799829) };
      \addlegendentry{\radixraduls};
      \addplot coordinates { (8192,220.131) (16384,196.484) (32768,95.6965) (65536,110.033) (131072,64.7041) (262144,35.2771) (524288,18.6066) (1048576,9.2921) (2097152,4.81232) (4194304,2.51897) (8388608,1.50638) (16777216,1.04002) (33554432,0.784934) (67108864,0.611853) (1.34218e+08,0.493202) (2.68435e+08,0.431246) (5.36871e+08,0.396638) (1.07374e+09,0.376244) (2.14748e+09,0.363688) (4.29497e+09,0.396147) (8.58993e+09,0.429752) (1.71799e+10,0.454957) };
      \addlegendentry{\radixregion};
      \addplot coordinates { (8192,42.5185) (16384,28.2235) (32768,22.3245) (65536,55.8322) (131072,25.3849) (262144,12.4183) (524288,6.09635) (1048576,3.15679) (2097152,1.88097) (4194304,1.1601) (8388608,0.802544) (16777216,0.673072) (33554432,0.592338) (67108864,0.526629) (1.34218e+08,0.493981) (2.68435e+08,0.475073) (5.36871e+08,0.467491) (1.07374e+09,0.467129) (2.14748e+09,0.457534) (4.29497e+09,0.43908) (8.58993e+09,0.428758) (1.71799e+10,0.421974) };
      \addlegendentry{\compiparassrsort};

      \nextgroupplot[
      xlabel={Item count $n$},
      title=\distuniform \pair,
      xmin=2^16,
      xmax=2^33,
      xtick={2^16, 2^20, 2^24, 2^28, 2^32},
      ]
      %% MULTIPLOT(algo|ptitle) select title as ptitle, size as x, 1000000.0 * milli * threads / (dsize * size * log(2, size)) as y, MULTIPLOT
      %% from pavgnames
      %% where algo not like 'mcstlmwm' and algo not like 'tbbparallelsort' and  algo not like 'ps4oparallel' and machine like 'i10pc132' and datatype like 'pair' and size >= 2^13
      %% and gen like 'random'
      %% and (algo not like 'tbbparallelsort' or 1000000.0 * milli * threads / (dsize * size * log(2, size)) <= 2)
      %% order by titleorder, x
      \addplot coordinates { (8192,14.3603) (16384,13.5317) (32768,20.1572) (65536,9.08011) (131072,6.63581) (262144,3.17646) (524288,2.81809) (1048576,1.68017) (2097152,0.986672) (4194304,0.790951) (8388608,0.621035) (16777216,0.526954) (33554432,0.470491) (67108864,0.445506) (1.34218e+08,0.428953) (2.68435e+08,0.412343) (5.36871e+08,0.401612) (1.07374e+09,0.38885) (2.14748e+09,0.386317) (4.29497e+09,0.389159) (8.58993e+09,0.395131) };
      \addlegendentry{\compiparassssort};
      \addplot coordinates { (8192,17.3779) (16384,14.5343) (32768,7.46832) (65536,3.63871) (131072,2.25603) (262144,1.62632) (524288,1.36714) (1048576,1.00828) (2097152,0.931624) (4194304,0.909495) (8388608,0.778432) (16777216,0.789316) (33554432,0.723002) (67108864,0.744019) (1.34218e+08,0.693526) (2.68435e+08,0.736328) (5.36871e+08,0.685595) (1.07374e+09,0.707747) (2.14748e+09,0.658409) (4.29497e+09,0.683189) (8.58993e+09,0.642534) };
      \addlegendentry{\compppbbs};
      \addplot coordinates { (8192,420.379) (16384,186.834) (32768,105.477) (65536,40.7332) (131072,153.362) (262144,83.2245) (524288,38.2158) (1048576,17.1639) (2097152,8.23093) (4194304,4.5036) (8388608,1.7414) (16777216,1.10756) (33554432,0.937941) (67108864,0.864145) (1.34218e+08,0.860597) (2.68435e+08,0.879103) (5.36871e+08,0.887809) (1.07374e+09,0.895895) (2.14748e+09,0.932729) (4.29497e+09,0.963249) (8.58993e+09,0.983539) };
      \addlegendentry{\comppbalancedsort};
      \addplot coordinates { (8192,20.9378) (16384,7.9694) (32768,4.41955) (65536,3.6762) (131072,1.95059) (262144,1.57909) (524288,1.04195) (1048576,0.865053) (2097152,0.625413) (4194304,0.603215) (8388608,0.632562) (16777216,0.563658) (33554432,0.545579) (67108864,0.533282) (1.34218e+08,0.519771) (2.68435e+08,0.538017) (5.36871e+08,0.544964) (1.07374e+09,0.516282) (2.14748e+09,0.506535) (4.29497e+09,0.526043) (8.58993e+09,0.531796) };
      \addlegendentry{\radixppbbr};
      \addplot coordinates { (8192,13570.2) (16384,6299.14) (32768,2949.17) (65536,1378.03) (131072,649.807) (262144,305.607) (524288,145.405) (1048576,68.4887) (2097152,33.3748) (4194304,16.4785) (8388608,8.48301) (16777216,4.73473) (33554432,2.95799) (67108864,2.10491) (1.34218e+08,1.68709) (2.68435e+08,1.46603) (5.36871e+08,1.33519) (1.07374e+09,1.22912) (2.14748e+09,1.16418) (4.29497e+09,1.14464) (8.58993e+09,1.13487) };
      \addlegendentry{\radixraduls};
      \addplot coordinates { (8192,95.3365) (16384,103.461) (32768,204.669) (65536,112.943) (131072,57.6382) (262144,29.0412) (524288,14.7161) (1048576,7.25481) (2097152,3.73039) (4194304,2.15458) (8388608,1.40337) (16777216,1.01463) (33554432,0.700087) (67108864,0.558007) (1.34218e+08,0.478058) (2.68435e+08,0.438086) (5.36871e+08,0.463655) (1.07374e+09,0.425335) (2.14748e+09,0.381139) (4.29497e+09,0.400873) (8.58993e+09,0.457775) };
      \addlegendentry{\radixregion};
      \addplot coordinates { (8192,12.8221) (16384,12.5783) (32768,47.0609) (65536,22.2166) (131072,10.5871) (262144,5.23023) (524288,2.61057) (1048576,1.44084) (2097152,0.861667) (4194304,0.64259) (8388608,0.575667) (16777216,0.484907) (33554432,0.430957) (67108864,0.409316) (1.34218e+08,0.391821) (2.68435e+08,0.377293) (5.36871e+08,0.371568) (1.07374e+09,0.367379) (2.14748e+09,0.399532) (4.29497e+09,0.394176) (8.58993e+09,0.382394) };
      \addlegendentry{\compiparassrsort};

      \legend{}
      
      \nextgroupplot[
      xlabel={Item count $n$},
      title=\distuniform \bytes,
      every legend/.append style={at=(ticklabel cs:1.1)},
      xmax=2^31.5,
      xmin=2^14,
      xmax=2^30,
      xtick={2^14, 2^18, 2^22, 2^26, 2^30},
      ]
      %% MULTIPLOT(algo|ptitle) select title as ptitle, size as x, 1000000.0 * milli * threads / (dsize * size * log(2, size)) as y, MULTIPLOT
      %% from pavgnames
      %% where algo not like 'mcstlmwm' and algo not like 'tbbparallelsort' and  algo not like 'ps4oparallel' and machine like 'i10pc132' and datatype like 'byte' and size >= 2^13
      %% and gen like 'byte'
      %% order by titleorder, x
      \addplot coordinates { (8192,14.9864) (16384,5.65129) (32768,3.38527) (65536,1.82339) (131072,1.24837) (262144,0.792706) (524288,0.725217) (1048576,0.629527) (2097152,0.511915) (4194304,0.442526) (8388608,0.430829) (16777216,0.418124) (33554432,0.399797) (67108864,0.383219) (1.34218e+08,0.379774) (2.68435e+08,0.360639) (5.36871e+08,0.364775) (1.07374e+09,0.376519) };
      \addlegendentry{\compiparassssort};
      \addplot coordinates { (8192,6.43081) (16384,2.35389) (32768,1.56522) (65536,1.06885) (131072,0.770487) (262144,0.611936) (524288,0.579634) (1048576,0.529464) (2097152,0.481036) (4194304,0.463006) (8388608,0.445715) (16777216,0.442247) (33554432,0.443449) (67108864,0.457279) (1.34218e+08,0.459904) (2.68435e+08,0.465743) (5.36871e+08,0.493317) (1.07374e+09,0.502916) };
      \addlegendentry{\compppbbs};
      \addplot coordinates { (8192,75.5174) (16384,35.973) (32768,12.181) (65536,9.17037) (131072,24.3873) (262144,14.4853) (524288,6.91873) (1048576,2.80114) (2097152,1.39166) (4194304,1.02789) (8388608,0.789581) (16777216,0.779996) (33554432,0.771305) (67108864,0.786817) (1.34218e+08,0.783104) (2.68435e+08,0.801121) (5.36871e+08,0.832681) (1.07374e+09,0.865662) };
      \addlegendentry{\comppbalancedsort};

      \legend{}
      
    \end{groupplot}
    \coordinate (c3) at ($(c1)!.5!(c2)$);
    \node[below] at (c3 |- current bounding box.south)
    {\pgfplotslegendfromname{legendplotp132}};
  \end{tikzpicture}

  \caption{
    Running times of parallel algorithms on different input distributions and data types of size $D$ executed on machine \pcinteltwo.
    A horizontal (vertical) line in the top right corner indicates that the algorithm's running time was too large for the plot (that the algorithm's interface does not accept the data type).
  }
  \label{fig:par rt distr types 132}
\end{figure}

\begin{figure}[tbp]
%% SQL
%% drop view if exists p CASCADE;
%% create view p as
%% select * from pradixalgoswithips4oml
%% union
%% select * from pcomparisonalgos

%% SQL
%% drop view if exists palgos1 CASCADE;
%% create view palgos1 as
%% select avgparallel.* from avgparallel inner join p
%% on avgparallel.algo = p.algo

%% SQL
%% drop view if exists palgos2 CASCADE;
%% create view palgos2 as
%% select * from palgos1 natural join pfast

%% SQL
%% drop view if exists palgos CASCADE;
%% create view palgos as
%% select machine, gen, datatype, algo, parallel, threads, vector, size, meminterleaved, copyback, milli
%% from palgos2

%% SQL
%% drop view if exists pavg CASCADE;
%% create view pavg as
%% select machine, algo, size, datatype, gen, threads, AVG(milli) as milli
%% from palgos
%% group by machine, algo, size, datatype, gen, threads


%% SQL
%% drop table if exists undisplayed CASCADE;
%% create table undisplayed(
%% machine character varying,
%% algo character varying,
%% size            bigint,
%% datatype character varying,
%% gen character varying,
%% threads            bigint,
%% milli  double precision
%% )

%% SQL
%% insert into undisplayed (machine, algo, datatype, gen, threads, size, milli)
%% values
%% ('i10pc133', 'pbbsradixsort', 'uint64', 'zipf', 1, -1, 0.1*2^17.5*17.5*8/1000000),
%% ('i10pc133', 'pbbsradixsort', 'uint64', 'zipf', 1, -2, 0.1*2^18.5*18.5*8/1000000),
%% ('i10pc133', 'pbbsradixsort', 'uint64', 'zipf', 1, -3, 0.1*2^19.5*19.5*8/1000000),
%% ('i10pc133', 'pbbsradixsort', 'double', 'random', 1, -1, 0.5*2^17*17*8/1000000),
%% ('i10pc133', 'pbbsradixsort', 'double', 'random', 1, -2, 0.3*2^17*17*8/1000000),
%% ('i10pc133', 'pbbsradixsort', 'double', 'random', 1, -3, 0.1*2^17*17*8/1000000),
%% ('i10pc133', 'raduls', 'double', 'random', 1, -1, 0.5*2^17.5*17.5*8/1000000),
%% ('i10pc133', 'raduls', 'double', 'random', 1, -2, 0.3*2^17.5*17.5*8/1000000),
%% ('i10pc133', 'raduls', 'double', 'random', 1, -3, 0.1*2^17.5*17.5*8/1000000),
%% ('i10pc133', 'regionsort', 'double', 'random', 1, -1, 0.5*2^18*18*8/1000000),
%% ('i10pc133', 'regionsort', 'double', 'random', 1, -2, 0.3*2^18*18*8/1000000),
%% ('i10pc133', 'regionsort', 'double', 'random', 1, -3, 0.1*2^18*18*8/1000000),
%% ('i10pc133', 'pbbsradixsort', 'byte', 'byte', 1, -1, 0.5*2^14.5*14.5*100/1000000),
%% ('i10pc133', 'pbbsradixsort', 'byte', 'byte', 1, -2, 0.3*2^14.5*14.5*100/1000000),
%% ('i10pc133', 'pbbsradixsort', 'byte', 'byte', 1, -3, 0.1*2^14.5*14.5*100/1000000),
%% ('i10pc133', 'raduls', 'byte', 'byte', 1, -1, 0.5*2^15*15*100/1000000),
%% ('i10pc133', 'raduls', 'byte', 'byte', 1, -2, 0.3*2^15*15*100/1000000),
%% ('i10pc133', 'raduls', 'byte', 'byte', 1, -3, 0.1*2^15*15*100/1000000),
%% ('i10pc133', 'regionsort', 'byte', 'byte', 1, -1, 0.5*2^15.5*15.5*100/1000000),
%% ('i10pc133', 'regionsort', 'byte', 'byte', 1, -2, 0.3*2^15.5*15.5*100/1000000),
%% ('i10pc133', 'regionsort', 'byte', 'byte', 1, -3, 0.1*2^15.5*15.5*100/1000000)

%% SQL
%% drop view if exists pavg1 CASCADE;
%% create view pavg1 as
%% select * from undisplayed union select * from pavg

%% SQL
%% drop view if exists pavgnames CASCADE;
%% create view pavgnames as
%% select pavg1.*, titles.title, titles.titleorder, datatypesizes.dsize from pavg1
%% inner join titles
%% on titles.algo like pavg1.algo
%% inner join datatypesizes
%% on pavg1.datatype = datatypesizes.datatype

  \begin{tikzpicture}
    \begin{groupplot}[
      group style={
        group size=2 by 4,
        vertical sep=1.2cm,
      },
      width=0.5\textwidth,
      height=0.22\textheight,
      xmode=log,
      log base x=2,
      plotstyleparallel,
      xmajorgrids=true,
      ymajorgrids=true,
      ymin=0,
      ymax=1.5,
      xmin=2^16,
      xmax=2^32,
      xtick={2^18, 2^22, 2^26, 2^30, 2^34},
      restrict y to domain=-1:30,
      ]
      
      
      \nextgroupplot[
      title=\distuniform \double]
      %% MULTIPLOT(algo|ptitle) select title as ptitle, size as x, 1000000.0 * milli * threads / (dsize * size * log(2, size)) as y, MULTIPLOT
      %% from pavgnames
      %% where algo not like 'aspasparallel' and algo not like 'gnuparallel' and algo not like 'tbbparallelsort' and  algo not like 'ps4oparallel' and machine like 'i10pc133' and datatype like 'double' and size >= 2^13
      %% and (algo not like 'tbbparallelsort' or 1000000.0 * milli * threads / (dsize * size * log(2, size)) <= 2)
      %% and gen like 'random'
      %% order by titleorder, x    
      \addplot coordinates { (8192,13.8705) (16384,13.2694) (32768,16.5151) (65536,13.6483) (131072,8.05483) (262144,2.35071) (524288,1.53443) (1048576,1.48542) (2097152,0.822588) (4194304,0.529764) (8388608,0.48673) (16777216,0.463521) (33554432,0.438794) (67108864,0.443463) (1.34218e+08,0.450096) (2.68435e+08,0.47283) (5.36871e+08,0.465568) (1.07374e+09,0.446005) (2.14748e+09,0.435313) (4.29497e+09,0.427707) };
      \addlegendentry{\compiparassssort};
      \addplot coordinates { (8192,8.64992) (16384,5.7386) (32768,3.96472) (65536,2.72621) (131072,2.01186) (262144,1.7531) (524288,1.35169) (1048576,0.997426) (2097152,0.882001) (4194304,0.892483) (8388608,0.906528) (16777216,0.825398) (33554432,0.803546) (67108864,0.773778) (1.34218e+08,0.775332) (2.68435e+08,0.750098) (5.36871e+08,0.747865) (1.07374e+09,0.728567) (2.14748e+09,0.729971) };
      \addlegendentry{\compppbbs};
      \addplot coordinates { (8192,279.952) (16384,130.156) (32768,68.7441) (65536,183.608) (131072,114.529) (262144,54.9833) (524288,25.7412) (1048576,10.587) (2097152,4.52012) (4194304,2.23154) (8388608,1.34753) (16777216,1.14683) (33554432,1.1031) (67108864,1.13037) (1.34218e+08,1.17438) (2.68435e+08,1.22936) (5.36871e+08,1.30883) (1.07374e+09,1.33692) (2.14748e+09,1.37479) (4.29497e+09,1.49011) };
      \addlegendentry{\comppbalancedsort};

      \legend{}
      
      \coordinate (c1) at (rel axis cs:0,1);
      
      \nextgroupplot[
      title=\distalmostsorted \ulong]
      %% MULTIPLOT(algo|ptitle) select title as ptitle, size as x, 1000000.0 * milli * threads / (dsize * size * log(2, size)) as y, MULTIPLOT
      %% from pavgnames
      %% where algo not like 'gnuparallel' and algo not like 'tbbparallelsort' and  algo not like 'ps4oparallel' and machine like 'i10pc133' and datatype like 'uint64' and size >= 2^13
      %% and gen like 'almostsorted'
      %% order by titleorder, x
      \addplot coordinates { (8192,6.75434) (16384,7.09295) (32768,27.4635) (65536,16.6666) (131072,5.78316) (262144,3.07211) (524288,1.42982) (1048576,0.796576) (2097152,0.562517) (4194304,0.423407) (8388608,0.302263) (16777216,0.262645) (33554432,0.24886) (67108864,0.280552) (1.34218e+08,0.346414) (2.68435e+08,0.361665) (5.36871e+08,0.344424) (1.07374e+09,0.318521) (2.14748e+09,0.288401) (4.29497e+09,0.265108) };
      \addlegendentry{\compiparassssort};
      \addplot coordinates { (8192,4.12041) (16384,4.56003) (32768,2.8108) (65536,1.65856) (131072,1.25647) (262144,0.945582) (524288,0.730467) (1048576,0.621097) (2097152,0.482251) (4194304,0.489123) (8388608,0.583133) (16777216,0.529855) (33554432,0.514789) (67108864,0.476148) (1.34218e+08,0.461717) (2.68435e+08,0.417573) (5.36871e+08,0.410603) (1.07374e+09,0.396762) (2.14748e+09,0.408057) };
      \addlegendentry{\compppbbs};
      \addplot coordinates { (8192,341.502) (16384,151.703) (32768,80.4878) (65536,181.196) (131072,94.1595) (262144,45.6321) (524288,18.6645) (1048576,9.90167) (2097152,3.84096) (4194304,1.87201) (8388608,0.937366) (16777216,0.653253) (33554432,0.580278) (67108864,0.579329) (1.34218e+08,0.60645) (2.68435e+08,0.636391) (5.36871e+08,0.656088) (1.07374e+09,0.689453) (2.14748e+09,0.712828) (4.29497e+09,0.741673) };
      \addlegendentry{\comppbalancedsort};
      \addplot coordinates { (8192,15.5677) (16384,8.96781) (32768,5.51059) (65536,3.65953) (131072,2.46478) (262144,2.05497) (524288,1.70147) (1048576,1.34098) (2097152,1.2861) (4194304,1.37468) (8388608,1.52328) (16777216,1.82674) (33554432,1.81451) (67108864,1.79138) (1.34218e+08,1.75624) (2.68435e+08,2.10762) (5.36871e+08,2.0506) (1.07374e+09,1.95324) };
      \addlegendentry{\radixppbbr};
      \addplot coordinates { (8192,1701.54) (16384,801.354) (32768,359.838) (65536,173.223) (131072,88.338) (262144,50.3224) (524288,32.3294) (1048576,22.3062) (2097152,18.0993) (4194304,3.29378) (8388608,2.13966) (16777216,1.53877) (33554432,1.3014) (67108864,1.16831) (1.34218e+08,1.08762) (2.68435e+08,1.0314) (5.36871e+08,0.986818) (1.07374e+09,0.912281) (2.14748e+09,0.881518) };
      \addlegendentry{\radixraduls};
      \addplot coordinates { (8192,13.0024) (16384,7.90543) (32768,10.8032) (65536,9.46395) (131072,2.20292) (262144,1.36898) (524288,0.92742) (1048576,0.719339) (2097152,0.391698) (4194304,0.347326) (8388608,0.258489) (16777216,0.240088) (33554432,0.328904) (67108864,0.334336) (1.34218e+08,0.335538) (2.68435e+08,0.313194) (5.36871e+08,0.301463) (1.07374e+09,0.292552) (2.14748e+09,0.282834) (4.29497e+09,0.285571) };
      \addlegendentry{\radixregion};
      \addplot coordinates { (8192,4.9585) (16384,3.03189) (32768,37.3384) (65536,3.7668) (131072,6.13604) (262144,2.66444) (524288,0.901597) (1048576,0.639983) (2097152,0.347179) (4194304,0.270163) (8388608,0.232115) (16777216,0.2677) (33554432,0.370784) (67108864,0.36215) (1.34218e+08,0.353029) (2.68435e+08,0.35415) (5.36871e+08,0.35986) (1.07374e+09,0.344881) (2.14748e+09,0.329808) (4.29497e+09,0.318466) };
      \addlegendentry{\compiparassrsort};

      \legend{}
      
      \coordinate (c2) at (rel axis cs:1,1);
      
      \nextgroupplot[
      y unit=ns,
      title=\distduplicatesroot \ulong]
      %% MULTIPLOT(algo|ptitle) select title as ptitle, size as x, 1000000.0 * milli * threads / (dsize * size * log(2, size)) as y, MULTIPLOT
      %% from pavgnames
      %% where algo not like 'gnuparallel' and algo not like 'tbbparallelsort' and  algo not like 'ps4oparallel' and machine like 'i10pc133' and datatype like 'uint64' and size >= 2^13
      %% and gen like 'rootdupls'
      %% order by titleorder, x
      \addplot coordinates { (8192,4.40569) (16384,5.5728) (32768,21.7038) (65536,13.526) (131072,8.03854) (262144,2.15886) (524288,2.30165) (1048576,0.818979) (2097152,0.559066) (4194304,0.303124) (8388608,0.352637) (16777216,0.372183) (33554432,0.356175) (67108864,0.382481) (1.34218e+08,0.39302) (2.68435e+08,0.388079) (5.36871e+08,0.377726) (1.07374e+09,0.363325) (2.14748e+09,0.367909) (4.29497e+09,0.372645) };
      \addlegendentry{\compiparassssort};
      \addplot coordinates { (8192,4.49443) (16384,4.01024) (32768,2.84102) (65536,1.69582) (131072,1.31846) (262144,0.875992) (524288,0.768879) (1048576,0.568583) (2097152,0.505961) (4194304,0.520367) (8388608,0.646013) (16777216,0.567144) (33554432,0.558416) (67108864,0.514252) (1.34218e+08,0.502203) (2.68435e+08,0.458899) (5.36871e+08,0.45229) (1.07374e+09,0.434066) (2.14748e+09,0.448173) };
      \addlegendentry{\compppbbs};
      \addplot coordinates { (8192,320.797) (16384,142.857) (32768,60.4599) (65536,468.745) (131072,214.438) (262144,98.3217) (524288,40.1659) (1048576,21.2907) (2097152,7.52926) (4194304,4.64552) (8388608,1.42994) (16777216,1.55138) (33554432,0.873362) (67108864,1.20289) (1.34218e+08,0.953627) (2.68435e+08,1.18204) (5.36871e+08,1.02052) (1.07374e+09,1.26173) (2.14748e+09,1.14246) (4.29497e+09,1.29843) };
      \addlegendentry{\comppbalancedsort};
      \addplot coordinates { (8192,14.377) (16384,9.85597) (32768,5.95085) (65536,4.04258) (131072,2.57134) (262144,2.16957) (524288,1.76881) (1048576,1.48166) (2097152,1.38437) (4194304,1.4251) (8388608,1.70716) (16777216,1.98967) (33554432,1.98149) (67108864,1.9921) (1.34218e+08,2.09875) (2.68435e+08,2.42949) (5.36871e+08,2.43982) (1.07374e+09,2.41064) };
      \addlegendentry{\radixppbbr};
      \addplot coordinates { (8192,1638.49) (16384,768.1) (32768,358.989) (65536,167.857) (131072,86.376) (262144,41.8769) (524288,22.9305) (1048576,12.4543) (2097152,8.10387) (4194304,5.49399) (8388608,3.77504) (16777216,2.5633) (33554432,2.0408) (67108864,1.73231) (1.34218e+08,1.47954) (2.68435e+08,1.08042) (5.36871e+08,1.03593) (1.07374e+09,0.992501) (2.14748e+09,0.958914) };
      \addlegendentry{\radixraduls};
      \addplot coordinates { (8192,8.1661) (16384,4.74638) (32768,37.9619) (65536,3.84516) (131072,3.53413) (262144,1.18819) (524288,1.09871) (1048576,0.477582) (2097152,0.445921) (4194304,0.447039) (8388608,0.404868) (16777216,0.438592) (33554432,0.444671) (67108864,0.496527) (1.34218e+08,0.484985) (2.68435e+08,0.469837) (5.36871e+08,0.454979) (1.07374e+09,0.436594) (2.14748e+09,0.420311) (4.29497e+09,0.511648) };
      \addlegendentry{\radixregion};
      \addplot coordinates { (8192,2.74569) (16384,1.69434) (32768,10.9316) (65536,16.5549) (131072,9.11206) (262144,5.82502) (524288,2.42319) (1048576,0.794407) (2097152,0.614863) (4194304,0.554696) (8388608,0.407654) (16777216,0.485378) (33554432,0.457793) (67108864,0.448559) (1.34218e+08,0.431597) (2.68435e+08,0.422286) (5.36871e+08,0.407701) (1.07374e+09,0.392345) (2.14748e+09,0.382434) (4.29497e+09,0.465606) };
      \addlegendentry{\compiparassrsort};

      \legend{}

      \nextgroupplot[
      title=\distduplicatestwice \ulong,
      ]
      %% MULTIPLOT(algo|ptitle) select title as ptitle, size as x, 1000000.0 * milli * threads / (dsize * size * log(2, size)) as y, MULTIPLOT
      %% from pavgnames
      %% where algo not like 'gnuparallel' and algo not like 'tbbparallelsort' and  algo not like 'ps4oparallel' and machine like 'i10pc133' and datatype like 'uint64' and size >= 2^13
      %% and (algo not like 'tbbparallelsort' or 1000000.0 * milli * threads / (dsize * size * log(2, size)) <= 2)
      %% and gen like 'twicedupes'
      %% order by titleorder, x
      \addplot coordinates { (8192,6.92758) (16384,10.2624) (32768,25.8411) (65536,16.5058) (131072,5.79311) (262144,3.636) (524288,1.67626) (1048576,1.12787) (2097152,0.800586) (4194304,0.486703) (8388608,0.465196) (16777216,0.415641) (33554432,0.401953) (67108864,0.407625) (1.34218e+08,0.457725) (2.68435e+08,0.475401) (5.36871e+08,0.464814) (1.07374e+09,0.445827) (2.14748e+09,0.43618) (4.29497e+09,0.426741) };
      \addlegendentry{\compiparassssort};
      \addplot coordinates { (8192,6.02596) (16384,5.08404) (32768,3.5378) (65536,2.5917) (131072,1.73898) (262144,1.33686) (524288,1.11052) (1048576,0.805899) (2097152,0.755245) (4194304,0.7347) (8388608,0.801842) (16777216,0.718875) (33554432,0.685457) (67108864,0.6528) (1.34218e+08,0.653572) (2.68435e+08,0.628246) (5.36871e+08,0.627829) (1.07374e+09,0.611644) (2.14748e+09,0.614571) };
      \addlegendentry{\compppbbs};
      \addplot coordinates { (8192,317.236) (16384,164.003) (32768,73.21) (65536,175.481) (131072,97.7845) (262144,47.3608) (524288,24.0798) (1048576,12.4551) (2097152,4.75397) (4194304,2.02595) (8388608,1.23336) (16777216,1.14013) (33554432,1.11146) (67108864,1.11423) (1.34218e+08,1.17291) (2.68435e+08,1.2345) (5.36871e+08,1.30117) (1.07374e+09,1.35178) (2.14748e+09,1.41044) (4.29497e+09,1.47434) };
      \addlegendentry{\comppbalancedsort};
      \addplot coordinates { (8192,18.0531) (16384,11.1225) (32768,6.61042) (65536,5.01437) (131072,3.86826) (262144,3.16311) (524288,2.39116) (1048576,2.02282) (2097152,1.72615) (4194304,1.95195) (8388608,2.08758) (16777216,2.3723) (33554432,2.24281) (67108864,2.19124) (1.34218e+08,2.15446) (2.68435e+08,2.77064) (5.36871e+08,2.57192) (1.07374e+09,2.7143) };
      \addlegendentry{\radixppbbr};
      \addplot coordinates { (8192,1721.58) (16384,807.477) (32768,374.171) (65536,176.425) (131072,90.3585) (262144,52.08) (524288,33.0125) (1048576,24.1326) (2097152,19.0037) (4194304,3.09118) (8388608,2.03855) (16777216,1.51529) (33554432,1.30508) (67108864,1.1811) (1.34218e+08,1.10756) (2.68435e+08,1.05374) (5.36871e+08,1.00957) (1.07374e+09,0.915095) (2.14748e+09,0.883176) };
      \addlegendentry{\radixraduls};
      \addplot coordinates { (8192,11.6262) (16384,7.01884) (32768,38.8536) (65536,36.4975) (131072,4.87924) (262144,2.6581) (524288,1.46375) (1048576,0.81925) (2097152,0.537801) (4194304,0.427135) (8388608,0.406105) (16777216,0.48485) (33554432,0.537145) (67108864,0.533484) (1.34218e+08,0.51865) (2.68435e+08,0.507105) (5.36871e+08,0.487558) (1.07374e+09,0.462265) (2.14748e+09,0.453736) (4.29497e+09,0.458315) };
      \addlegendentry{\radixregion};
      \addplot coordinates { (8192,3.97611) (16384,3.06368) (32768,14.4864) (65536,7.95817) (131072,7.57249) (262144,2.44539) (524288,2.38673) (1048576,0.9939) (2097152,0.629319) (4194304,0.339936) (8388608,0.34505) (16777216,0.388901) (33554432,0.539169) (67108864,0.543468) (1.34218e+08,0.531723) (2.68435e+08,0.520498) (5.36871e+08,0.494667) (1.07374e+09,0.478186) (2.14748e+09,0.46435) (4.29497e+09,0.452375) };
      \addlegendentry{\compiparassrsort};

      \legend{}
      
      \nextgroupplot[
      every axis y label/.append style={at=(ticklabel cs:1.1)},
      ylabel={Running time $t/ Dn \log_2 n$},
      title=\distexpo \ulong]
      %% MULTIPLOT(algo|ptitle) select title as ptitle, size as x, 1000000.0 * milli * threads / (dsize * size * log(2, size)) as y, MULTIPLOT
      %% from pavgnames
      %% where algo not like 'gnuparallel' and algo not like 'tbbparallelsort' and  algo not like 'ps4oparallel' and machine like 'i10pc133' and datatype like 'uint64' and size >= 2^13
      %% and (algo not like 'tbbparallelsort' or 1000000.0 * milli * threads / (dsize * size * log(2, size)) <= 2)
      %% and gen like 'exponential'
      %% order by titleorder, x
      \addplot coordinates { (8192,5.49761) (16384,8.22906) (32768,18.4387) (65536,14.104) (131072,6.31672) (262144,4.49949) (524288,2.09012) (1048576,1.38453) (2097152,1.07839) (4194304,0.707702) (8388608,0.503157) (16777216,0.42671) (33554432,0.375524) (67108864,0.385316) (1.34218e+08,0.418257) (2.68435e+08,0.428888) (5.36871e+08,0.421064) (1.07374e+09,0.398116) (2.14748e+09,0.392189) (4.29497e+09,0.383618) };
      \addlegendentry{\compiparassssort};
      \addplot coordinates { (8192,7.23996) (16384,4.87467) (32768,3.26332) (65536,2.40912) (131072,1.81111) (262144,1.28398) (524288,1.00512) (1048576,0.739254) (2097152,0.640695) (4194304,0.679138) (8388608,0.740663) (16777216,0.653043) (33554432,0.621697) (67108864,0.581452) (1.34218e+08,0.572439) (2.68435e+08,0.534931) (5.36871e+08,0.534156) (1.07374e+09,0.51501) (2.14748e+09,0.525967) };
      \addlegendentry{\compppbbs};
      \addplot coordinates { (8192,250.312) (16384,154.096) (32768,60.6135) (65536,207.852) (131072,96.8) (262144,48.4723) (524288,23.5579) (1048576,10.7936) (2097152,5.17679) (4194304,2.36095) (8388608,1.19201) (16777216,1.03257) (33554432,1.01821) (67108864,1.065) (1.34218e+08,1.12976) (2.68435e+08,1.15805) (5.36871e+08,1.20756) (1.07374e+09,1.2332) (2.14748e+09,1.26746) (4.29497e+09,1.31665) };
      \addlegendentry{\comppbalancedsort};
      \addplot coordinates { (8192,5.14) (16384,3.44304) (32768,2.23485) (65536,1.55313) (131072,1.07578) (262144,0.836368) (524288,0.619964) (1048576,0.538785) (2097152,0.462821) (4194304,0.515373) (8388608,0.561867) (16777216,0.54403) (33554432,0.593921) (67108864,0.684245) (1.34218e+08,0.669186) (2.68435e+08,0.698604) (5.36871e+08,0.712474) (1.07374e+09,0.713407) (2.14748e+09,0.699652) };
      \addlegendentry{\radixppbbr};
      \addplot coordinates { (8192,692.893) (16384,508.95) (32768,231.337) (65536,106.189) (131072,45.8863) (262144,23.2241) (524288,11.2546) (1048576,5.77495) (2097152,2.60969) (4194304,1.43405) (8388608,1.06606) (16777216,0.931482) (33554432,0.831624) (67108864,0.787603) (1.34218e+08,0.712855) (2.68435e+08,0.666679) (5.36871e+08,0.657422) (1.07374e+09,0.661821) (2.14748e+09,0.641819) };
      \addlegendentry{\radixraduls};
      \addplot coordinates { (8192,38.4017) (16384,41.3776) (32768,73.622) (65536,39.5162) (131072,20.0883) (262144,10.1466) (524288,5.07768) (1048576,2.66615) (2097152,1.4439) (4194304,0.931841) (8388608,0.681911) (16777216,0.559398) (33554432,0.584836) (67108864,0.551633) (1.34218e+08,0.536626) (2.68435e+08,0.538546) (5.36871e+08,0.546077) (1.07374e+09,0.544651) (2.14748e+09,0.529009) (4.29497e+09,0.514291) };
      \addlegendentry{\radixregion};
      \addplot coordinates { (8192,8.49584) (16384,7.60644) (32768,20.5137) (65536,14.2571) (131072,4.67262) (262144,2.12881) (524288,1.3785) (1048576,1.34405) (2097152,0.457159) (4194304,0.380372) (8388608,0.386105) (16777216,0.396082) (33554432,0.434449) (67108864,0.470402) (1.34218e+08,0.543292) (2.68435e+08,0.567382) (5.36871e+08,0.560669) (1.07374e+09,0.556579) (2.14748e+09,0.541987) (4.29497e+09,0.532749) };
      \addlegendentry{\compiparassrsort};

      \legend{}
      
      \nextgroupplot[
      title=\distzipf \ulong,
      legend to name=legendplotp133,
      legend style={at={($(0,0)+(1cm,1cm)$)},legend columns=5,fill=none,draw=black,anchor=center,align=center},
      ]
      %% MULTIPLOT(algo|ptitle) select title as ptitle, size as x, 1000000.0 * milli * threads / (dsize * size * log(2, size)) as y, MULTIPLOT
      %% from pavgnames
      %% where algo not like 'gnuparallel' and algo not like 'tbbparallelsort' and  algo not like 'ps4oparallel' and machine like 'i10pc133' and datatype like 'uint64' and size >= 2^13
      %% and (algo not like 'tbbparallelsort' or 1000000.0 * milli * threads / (dsize * size * log(2, size)) <= 2)
      %% and gen like 'zipf'
      %% order by titleorder, x
      \addplot coordinates { (8192,10.1999) (16384,11.4603) (32768,23.0117) (65536,12.245) (131072,7.68362) (262144,3.92413) (524288,2.4838) (1048576,1.10749) (2097152,0.820054) (4194304,0.588629) (8388608,0.403002) (16777216,0.3844) (33554432,0.384162) (67108864,0.398171) (1.34218e+08,0.447137) (2.68435e+08,0.455452) (5.36871e+08,0.442397) (1.07374e+09,0.423034) (2.14748e+09,0.411973) (4.29497e+09,0.401834) };
      \addlegendentry{\compiparassssort};
      \addplot coordinates { (8192,8.31308) (16384,5.19372) (32768,3.58995) (65536,2.52922) (131072,2.0155) (262144,1.66243) (524288,1.19506) (1048576,0.821057) (2097152,0.757298) (4194304,0.736677) (8388608,0.761922) (16777216,0.692975) (33554432,0.65598) (67108864,0.621661) (1.34218e+08,0.603159) (2.68435e+08,0.563757) (5.36871e+08,0.553875) (1.07374e+09,0.541118) (2.14748e+09,0.550416) };
      \addlegendentry{\compppbbs};
      \addplot coordinates { (8192,312.386) (16384,128.324) (32768,75.0628) (65536,166.144) (131072,91.9715) (262144,50.2041) (524288,24.7548) (1048576,10.7903) (2097152,3.93705) (4194304,2.33928) (8388608,1.28657) (16777216,1.08258) (33554432,1.04002) (67108864,1.09189) (1.34218e+08,1.14492) (2.68435e+08,1.2089) (5.36871e+08,1.27334) (1.07374e+09,1.29722) (2.14748e+09,1.38167) (4.29497e+09,1.42417) };
      \addlegendentry{\comppbalancedsort};
      \addplot coordinates { (8192,18.303) (16384,17.7536) (32768,7.88302) (65536,6.6019) (131072,5.05342) (262144,4.15069) (524288,3.24022) (1048576,3.25151) (2097152,2.70397) (4194304,2.76329) (8388608,2.72721) (16777216,2.88186) (33554432,2.84363) (67108864,2.77985) (1.34218e+08,2.73313) (2.68435e+08,3.09747) (5.36871e+08,3.04732) (1.07374e+09,2.97115) };
      \addlegendentry{\radixppbbr};
      \addplot coordinates { (8192,2483.49) (16384,1678.39) (32768,835.607) (65536,409.134) (131072,188.184) (262144,89.7284) (524288,43.6681) (1048576,21.7671) (2097152,11.9217) (4194304,6.58869) (8388608,3.82592) (16777216,2.45397) (33554432,1.79508) (67108864,1.43795) (1.34218e+08,1.22928) (2.68435e+08,1.11273) (5.36871e+08,1.04189) (1.07374e+09,0.994905) (2.14748e+09,0.960392) };
      \addlegendentry{\radixraduls};
      \addplot coordinates { (8192,44.2181) (16384,28.7108) (32768,19.1932) (65536,22.3044) (131072,11.9656) (262144,6.14513) (524288,3.31129) (1048576,1.82161) (2097152,1.06302) (4194304,0.711345) (8388608,0.591162) (16777216,0.587175) (33554432,0.573839) (67108864,0.549414) (1.34218e+08,0.537231) (2.68435e+08,0.526006) (5.36871e+08,0.508219) (1.07374e+09,0.491773) (2.14748e+09,0.484112) (4.29497e+09,0.506217) };
      \addlegendentry{\radixregion};
      \addplot coordinates { (8192,7.79352) (16384,4.65882) (32768,14.938) (65536,10.4592) (131072,8.49603) (262144,2.78287) (524288,2.07811) (1048576,0.816748) (2097152,0.653528) (4194304,0.446239) (8388608,0.431978) (16777216,0.559663) (33554432,0.574661) (67108864,0.553918) (1.34218e+08,0.538709) (2.68435e+08,0.534403) (5.36871e+08,0.525146) (1.07374e+09,0.519272) (2.14748e+09,0.515513) (4.29497e+09,0.507964) };
      \addlegendentry{\compiparassrsort};

      \nextgroupplot[
      xlabel={Item count $n$},
      title=\distuniform \pair,
      xmax=2^34.5,
      xmin=2^15,
      xmax=2^31,
      xtick={2^16, 2^20, 2^24, 2^28, 2^32},
      ]
      %% MULTIPLOT(algo|ptitle) select title as ptitle, size as x, 1000000.0 * milli * threads / (dsize * size * log(2, size)) as y, MULTIPLOT
      %% from pavgnames
      %% where algo not like 'gnuparallel' and algo not like 'tbbparallelsort' and  algo not like 'ps4oparallel' and machine like 'i10pc133' and datatype like 'pair' and size >= 2^13
      %% and gen like 'random'
      %% order by titleorder, x
      \addplot coordinates { (8192,7.48722) (16384,26.687) (32768,14.1745) (65536,5.75335) (131072,5.05943) (262144,1.36552) (524288,1.15928) (1048576,0.653595) (2097152,0.540309) (4194304,0.402972) (8388608,0.378523) (16777216,0.376356) (33554432,0.407563) (67108864,0.464268) (1.34218e+08,0.48687) (2.68435e+08,0.476726) (5.36871e+08,0.460758) (1.07374e+09,0.442934) (2.14748e+09,0.433362) };
      \addlegendentry{\compiparassssort};
      \addplot coordinates { (8192,3.55196) (16384,4.36094) (32768,2.41128) (65536,1.66592) (131072,1.27446) (262144,0.944277) (524288,0.717256) (1048576,0.600546) (2097152,0.612789) (4194304,0.772084) (8388608,0.676606) (16777216,0.698403) (33554432,0.621488) (67108864,0.645726) (1.34218e+08,0.569803) (2.68435e+08,0.57228) (5.36871e+08,0.509385) (1.07374e+09,0.534499) };
      \addlegendentry{\compppbbs};
      \addplot coordinates { (8192,183.643) (16384,81.8107) (32768,34.2376) (65536,96.8904) (131072,57.4252) (262144,25.271) (524288,11.9426) (1048576,5.13371) (2097152,2.5551) (4194304,1.25636) (8388608,1.12302) (16777216,1.01821) (33554432,1.05336) (67108864,1.12049) (1.34218e+08,1.16876) (2.68435e+08,1.22776) (5.36871e+08,1.2997) (1.07374e+09,1.33205) (2.14748e+09,1.39545) };
      \addlegendentry{\comppbalancedsort};
      \addplot coordinates { (8192,3.24451) (16384,1.95716) (32768,1.28819) (65536,0.960083) (131072,0.709639) (262144,0.603262) (524288,0.547089) (1048576,0.447202) (2097152,0.496746) (4194304,0.505155) (8388608,0.486447) (16777216,0.472149) (33554432,0.460525) (67108864,0.442145) (1.34218e+08,0.43658) (2.68435e+08,0.43766) (5.36871e+08,0.458008) (1.07374e+09,0.443358) };
      \addlegendentry{\radixppbbr};
      \addplot coordinates { (8192,1095.89) (16384,523.069) (32768,243.712) (65536,111.363) (131072,52.6516) (262144,27.2445) (524288,14.2809) (1048576,8.24513) (2097152,5.12863) (4194304,3.25956) (8388608,2.32925) (16777216,1.92811) (33554432,1.73252) (67108864,1.65675) (1.34218e+08,1.56762) (2.68435e+08,1.53266) (5.36871e+08,1.47546) (1.07374e+09,1.39302) };
      \addlegendentry{\radixraduls};
      \addplot coordinates { (8192,15.5762) (16384,12.946) (32768,36.058) (65536,18.9202) (131072,9.86329) (262144,5.02718) (524288,2.59618) (1048576,1.41153) (2097152,0.899396) (4194304,0.624152) (8388608,0.510001) (16777216,0.663008) (33554432,0.572687) (67108864,0.554017) (1.34218e+08,0.535697) (2.68435e+08,0.518686) (5.36871e+08,0.508234) (1.07374e+09,0.490842) (2.14748e+09,0.473106) };
      \addlegendentry{\radixregion};
      \addplot coordinates { (8192,5.08709) (16384,22.9921) (32768,11.8164) (65536,5.08593) (131072,4.43026) (262144,1.47648) (524288,1.02892) (1048576,0.549766) (2097152,0.331609) (4194304,0.33733) (8388608,0.376611) (16777216,0.382136) (33554432,0.433242) (67108864,0.502615) (1.34218e+08,0.499572) (2.68435e+08,0.481417) (5.36871e+08,0.461282) (1.07374e+09,0.44497) (2.14748e+09,0.431692) };
      \addlegendentry{\compiparassrsort};

      \legend{}
      
      \nextgroupplot[
      xlabel={Item count $n$},
      title=\distuniform \bytes,
      every legend/.append style={at=(ticklabel cs:1.1)},
      xmax=2^31.5,
      xmin=2^13,
      xmax=2^29,
      xtick={2^14, 2^18, 2^22, 2^26, 2^30},
      ]
      %% MULTIPLOT(algo|ptitle) select title as ptitle, size as x, 1000000.0 * milli * threads / (dsize * size * log(2, size)) as y, MULTIPLOT
      %% from pavgnames
      %% where algo not like 'gnuparallel' and algo not like 'tbbparallelsort' and  algo not like 'ps4oparallel' and machine like 'i10pc133' and datatype like 'byte' and size >= 2^13
      %% and gen like 'byte'
      %% order by titleorder, x
      \addplot coordinates { (8192,14.0881) (16384,4.97137) (32768,2.50283) (65536,1.85369) (131072,0.53208) (262144,0.465379) (524288,0.399103) (1048576,0.402502) (2097152,0.410348) (4194304,0.434117) (8388608,0.495045) (16777216,0.527828) (33554432,0.523828) (67108864,0.505143) (1.34218e+08,0.488151) (2.68435e+08,0.474078) (5.36871e+08,0.470727) };
      \addlegendentry{\compiparassssort};
      \addplot coordinates { (8192,2.39096) (16384,1.14102) (32768,0.811361) (65536,0.626549) (131072,0.524187) (262144,0.475494) (524288,0.576116) (1048576,0.580999) (2097152,0.561466) (4194304,0.542087) (8388608,0.490727) (16777216,0.475242) (33554432,0.459858) (67108864,0.469543) (1.34218e+08,0.489167) };
      \addlegendentry{\compppbbs};
      \addplot coordinates { (8192,28.1048) (16384,10.7014) (32768,6.72228) (65536,13.2884) (131072,8.53432) (262144,4.37185) (524288,1.88977) (1048576,1.12802) (2097152,1.04121) (4194304,1.00932) (8388608,1.02842) (16777216,1.05663) (33554432,1.10434) (67108864,1.14344) (1.34218e+08,1.22622) (2.68435e+08,1.2273) };
      \addlegendentry{\comppbalancedsort};

      \legend{}
      
    \end{groupplot}
    \coordinate (c3) at ($(c1)!.5!(c2)$);
    \node[below] at (c3 |- current bounding box.south)
    {\pgfplotslegendfromname{legendplotp133}};
  \end{tikzpicture}

  \caption{
    Running times of parallel algorithms on different input distributions and data types of size $D$ executed on machine \pcamd.
    A horizontal (vertical) line in the top right corner indicates that the algorithm's running time was too large for the plot (that the algorithm's interface does not accept the data type).
  }
  \label{fig:par rt distr types 133}
\end{figure}

\begin{figure}[tbp]
%% SQL
%% drop view if exists p CASCADE;
%% create view p as
%% select * from pradixalgoswithips4oml
%% union
%% select * from pcomparisonalgos

%% SQL
%% drop view if exists palgos1 CASCADE;
%% create view palgos1 as
%% select avgparallel.* from avgparallel inner join p
%% on avgparallel.algo = p.algo

%% SQL
%% drop view if exists palgos2 CASCADE;
%% create view palgos2 as
%% select * from palgos1 natural join pfast

%% SQL
%% drop view if exists palgos CASCADE;
%% create view palgos as
%% select machine, gen, datatype, algo, parallel, threads, vector, size, meminterleaved, copyback, milli
%% from palgos2

%% SQL
%% drop view if exists pavg CASCADE;
%% create view pavg as
%% select machine, algo, size, datatype, gen, threads, AVG(milli) as milli
%% from palgos
%% group by machine, algo, size, datatype, gen, threads


%% SQL
%% drop table if exists undisplayed CASCADE;
%% create table undisplayed(
%% machine character varying,
%% algo character varying,
%% size            bigint,
%% datatype character varying,
%% gen character varying,
%% threads            bigint,
%% milli  double precision
%% )

%% SQL
%% insert into undisplayed (machine, algo, datatype, gen, threads, size, milli)
%% values
%% ('i10pc135', 'pbbsradixsort', 'uint64', 'twicedupes', 1, -1, 0.3*2^19.5*19.5*8/1000000),
%% ('i10pc135', 'pbbsradixsort', 'uint64', 'twicedupes', 1, -2, 0.3*2^20.5*20.5*8/1000000),
%% ('i10pc135', 'pbbsradixsort', 'uint64', 'twicedupes', 1, -3, 0.3*2^21.5*21.5*8/1000000),
%% ('i10pc135', 'pbbsradixsort', 'uint64', 'almostsorted', 1, -1, 0.3*2^19.5*19.5*8/1000000),
%% ('i10pc135', 'pbbsradixsort', 'uint64', 'almostsorted', 1, -2, 0.3*2^20.5*20.5*8/1000000),
%% ('i10pc135', 'pbbsradixsort', 'uint64', 'almostsorted', 1, -3, 0.3*2^21.5*21.5*8/1000000),
%% ('i10pc135', 'pbbsradixsort', 'uint64', 'rootdupls', 1, -1, 0.3*2^19.5*19.5*8/1000000),
%% ('i10pc135', 'pbbsradixsort', 'uint64', 'rootdupls', 1, -2, 0.3*2^20.5*20.5*8/1000000),
%% ('i10pc135', 'pbbsradixsort', 'uint64', 'rootdupls', 1, -3, 0.3*2^21.5*21.5*8/1000000),
%% ('i10pc135', 'pbbsradixsort', 'uint64', 'zipf', 1, -1, 0.3*2^19.5*19.5*8/1000000),
%% ('i10pc135', 'pbbsradixsort', 'uint64', 'zipf', 1, -2, 0.3*2^20.5*20.5*8/1000000),
%% ('i10pc135', 'pbbsradixsort', 'uint64', 'zipf', 1, -3, 0.3*2^21.5*21.5*8/1000000),
%% ('i10pc135', 'raduls', 'pair', 'random', 1, -1, 0.3*2^17.5*17.5*16/1000000),
%% ('i10pc135', 'raduls', 'pair', 'random', 1, -2, 0.3*2^18.5*18.5*16/1000000),
%% ('i10pc135', 'raduls', 'pair', 'random', 1, -3, 0.3*2^19.5*19.5*16/1000000),
%% ('i10pc135', 'pbbsradixsort', 'double', 'random', 1, -1, 0.5*2^19*19*8/1000000),
%% ('i10pc135', 'pbbsradixsort', 'double', 'random', 1, -2, 0.3*2^19*19*8/1000000),
%% ('i10pc135', 'pbbsradixsort', 'double', 'random', 1, -3, 0.1*2^19*19*8/1000000),
%% ('i10pc135', 'raduls', 'double', 'random', 1, -1, 0.5*2^19.5*19.5*8/1000000),
%% ('i10pc135', 'raduls', 'double', 'random', 1, -2, 0.3*2^19.5*19.5*8/1000000),
%% ('i10pc135', 'raduls', 'double', 'random', 1, -3, 0.1*2^19.5*19.5*8/1000000),
%% ('i10pc135', 'regionsort', 'double', 'random', 1, -1, 0.5*2^20*20*8/1000000),
%% ('i10pc135', 'regionsort', 'double', 'random', 1, -2, 0.3*2^20*20*8/1000000),
%% ('i10pc135', 'regionsort', 'double', 'random', 1, -3, 0.1*2^20*20*8/1000000),
%% ('i10pc135', 'gnuquicksortunbalanced', 'byte', 'byte', 1, -1, 0.5*2^17*17*100/1000000),
%% ('i10pc135', 'gnuquicksortunbalanced', 'byte', 'byte', 1, -2, 0.5*2^18*18*100/1000000),
%% ('i10pc135', 'gnuquicksortunbalanced', 'byte', 'byte', 1, -3, 0.5*2^19*19*100/1000000),
%% ('i10pc135', 'gnuquicksortbalanced', 'byte', 'byte', 1, -1, 0.3*2^17*17*100/1000000),
%% ('i10pc135', 'gnuquicksortbalanced', 'byte', 'byte', 1, -2, 0.3*2^18*18*100/1000000),
%% ('i10pc135', 'gnuquicksortbalanced', 'byte', 'byte', 1, -3, 0.3*2^19*19*100/1000000),
%% ('i10pc135', 'pbbsradixsort', 'byte', 'byte', 1, -1, 0.5*2^15.5*15.5*100/1000000),
%% ('i10pc135', 'pbbsradixsort', 'byte', 'byte', 1, -2, 0.3*2^15.5*15.5*100/1000000),
%% ('i10pc135', 'pbbsradixsort', 'byte', 'byte', 1, -3, 0.1*2^15.5*15.5*100/1000000),
%% ('i10pc135', 'raduls', 'byte', 'byte', 1, -1, 0.5*2^16*16*100/1000000),
%% ('i10pc135', 'raduls', 'byte', 'byte', 1, -2, 0.3*2^16*16*100/1000000),
%% ('i10pc135', 'raduls', 'byte', 'byte', 1, -3, 0.1*2^16*16*100/1000000),
%% ('i10pc135', 'regionsort', 'byte', 'byte', 1, -1, 0.5*2^16.5*16.5*100/1000000),
%% ('i10pc135', 'regionsort', 'byte', 'byte', 1, -2, 0.3*2^16.5*16.5*100/1000000),
%% ('i10pc135', 'regionsort', 'byte', 'byte', 1, -3, 0.1*2^16.5*16.5*100/1000000)

%% SQL
%% drop view if exists pavg1 CASCADE;
%% create view pavg1 as
%% select * from undisplayed union select * from pavg

%% SQL
%% drop view if exists pavgnames CASCADE;
%% create view pavgnames as
%% select pavg1.*, titles.title, titles.titleorder, datatypesizes.dsize from pavg1
%% inner join titles
%% on titles.algo like pavg1.algo
%% inner join datatypesizes
%% on pavg1.datatype = datatypesizes.datatype

  \begin{tikzpicture}
    \begin{groupplot}[
      group style={
        group size=2 by 4,
        vertical sep=1.2cm,
      },
      width=0.5\textwidth,
      height=0.22\textheight,
      xmode=log,
      log base x=2,
      plotstyleparallel,
      xmajorgrids=true,
      ymajorgrids=true,
      ymin=0,
      ymax=1.5,
      xmin=2^18,
      xmax=2^34,
      xtick={2^18, 2^22, 2^26, 2^30, 2^34},
      restrict y to domain=-1:30,
      ]
      
      
      \nextgroupplot[
      title=\distuniform \double]
      %% MULTIPLOT(algo|ptitle) select title as ptitle, size as x, 1000000.0 * milli * threads / (dsize * size * log(2, size)) as y, MULTIPLOT
      %% from pavgnames
      %% where algo not like 'aspasparallel' and algo not like 'gnuparallel' and algo not like 'tbbparallelsort' and  algo not like 'ps4oparallel' and machine like 'i10pc135' and datatype like 'double' and size >= 2^13
      %% and (algo not like 'tbbparallelsort' or 1000000.0 * milli * threads / (dsize * size * log(2, size)) <= 2)
      %% and gen like 'random'
      %% order by titleorder, x    
      \addplot coordinates { (8192,89.9119) (16384,86.7329) (32768,74.5247) (65536,89.7521) (131072,79.0446) (262144,136.905) (524288,75.7827) (1048576,40.3534) (2097152,22.9274) (4194304,8.80045) (8388608,5.92712) (16777216,2.71442) (33554432,1.45303) (67108864,1.003) (1.34218e+08,0.869835) (2.68435e+08,0.632931) (5.36871e+08,0.562467) (1.07374e+09,0.557513) (2.14748e+09,0.523304) (4.29497e+09,0.509282) (8.58993e+09,0.509512) (1.71799e+10,0.503585) };
      \addlegendentry{\compiparassssort};
      \addplot coordinates { (8192,282.579) (16384,186.481) (32768,76.8438) (65536,35.4792) (131072,22.1373) (262144,10.464) (524288,7.16096) (1048576,4.79911) (2097152,3.70591) (4194304,2.93558) (8388608,2.23867) (16777216,1.88618) (33554432,1.62077) (67108864,1.48652) (1.34218e+08,1.25996) (2.68435e+08,1.16998) (5.36871e+08,1.27993) (1.07374e+09,1.20971) (2.14748e+09,1.295) (4.29497e+09,1.22969) (8.58993e+09,1.29713) (1.71799e+10,1.23413) };
      \addlegendentry{\compppbbs};
      \addplot coordinates { (8192,6002.48) (16384,2622.56) (32768,1360.36) (65536,569.342) (131072,314.115) (262144,180.375) (524288,675.837) (1048576,297.115) (2097152,147.815) (4194304,71.2229) (8388608,32.7423) (16777216,15.7232) (33554432,7.86887) (67108864,4.67049) (1.34218e+08,3.29772) (2.68435e+08,2.3438) (5.36871e+08,2.1358) (1.07374e+09,1.92999) (2.14748e+09,2.28174) (4.29497e+09,1.71878) (8.58993e+09,1.7903) (1.71799e+10,1.71437) };
      \addlegendentry{\comppbalancedsort};

      \legend{}
      
      \coordinate (c1) at (rel axis cs:0,1);
      
      \nextgroupplot[
      title=\distalmostsorted \ulong]
      %% MULTIPLOT(algo|ptitle) select title as ptitle, size as x, 1000000.0 * milli * threads / (dsize * size * log(2, size)) as y, MULTIPLOT
      %% from pavgnames
      %% where algo not like 'gnuparallel' and algo not like 'tbbparallelsort' and  algo not like 'ps4oparallel' and machine like 'i10pc135' and datatype like 'uint64' and size >= 2^13
      %% and gen like 'almostsorted'
      %% and (algo not like 'tbbparallelsort' or 1000000.0 * milli * threads / (dsize * size * log(2, size)) <= 2)
      %% order by titleorder, x
      \addplot coordinates { (8192,62.3745) (16384,64.5966) (32768,59.0134) (65536,57.998) (131072,52.7584) (262144,251.272) (524288,60.3294) (1048576,21.348) (2097152,13.8522) (4194304,7.75966) (8388608,3.36266) (16777216,1.98786) (33554432,0.995428) (67108864,0.693574) (1.34218e+08,0.571736) (2.68435e+08,0.506366) (5.36871e+08,0.415575) (1.07374e+09,0.395417) (2.14748e+09,0.321622) (4.29497e+09,0.342089) (8.58993e+09,0.325594) (1.71799e+10,0.317365) };
      \addlegendentry{\compiparassssort};
      \addplot coordinates { (8192,124.971) (16384,210.77) (32768,84.7008) (65536,39.0579) (131072,15.4409) (262144,9.48915) (524288,4.92006) (1048576,3.14548) (2097152,2.34667) (4194304,1.64654) (8388608,1.27048) (16777216,0.975591) (33554432,0.783699) (67108864,0.705829) (1.34218e+08,0.688316) (2.68435e+08,0.592315) (5.36871e+08,0.682861) (1.07374e+09,1.15526) (2.14748e+09,1.46692) (4.29497e+09,2.31063) (8.58993e+09,2.22115) (1.71799e+10,2.53888) };
      \addlegendentry{\compppbbs};
      \addplot coordinates { (8192,5724.27) (16384,2701.54) (32768,1093.81) (65536,571.001) (131072,293.541) (262144,183.625) (524288,648.839) (1048576,298.715) (2097152,145.072) (4194304,67.6893) (8388608,29.1533) (16777216,12.9015) (33554432,6.37665) (67108864,3.57116) (1.34218e+08,1.91129) (2.68435e+08,1.20314) (5.36871e+08,0.939134) (1.07374e+09,0.878715) (2.14748e+09,0.851947) (4.29497e+09,0.824009) (8.58993e+09,0.878385) (1.71799e+10,0.85568) };
      \addlegendentry{\comppbalancedsort};
      \addplot coordinates { (8192,139.912) (16384,176.766) (32768,136.399) (65536,70.8371) (131072,41.9578) (262144,29.0521) (524288,17.6838) (1048576,12.8912) (2097152,9.2268) (4194304,8.10282) (8388608,8.10535) (16777216,7.85581) (33554432,7.80697) (67108864,6.63955) (1.34218e+08,6.3661) (2.68435e+08,7.34431) (5.36871e+08,7.13279) (1.07374e+09,5.26782) (2.14748e+09,3.86856) (4.29497e+09,7.79773) (8.58993e+09,7.36838) };
      \addlegendentry{\radixppbbr};
      \addplot coordinates { (8192,217124) (16384,100759) (32768,47170.8) (65536,22161.9) (131072,11969.4) (262144,6977.77) (524288,4674.04) (1048576,3457.77) (2097152,2778.22) (4194304,191.212) (8388608,90.6879) (16777216,44.1108) (33554432,24.7963) (67108864,15.8624) (1.34218e+08,11.4838) (2.68435e+08,9.36189) (5.36871e+08,7.84873) (1.07374e+09,2.38815) (2.14748e+09,2.28727) (4.29497e+09,1.89067) (8.58993e+09,2.04548) (1.71799e+10,1.69489) };
      \addlegendentry{\radixraduls};
      \addplot coordinates { (8192,268.639) (16384,157.545) (32768,259.004) (65536,181.484) (131072,52.5123) (262144,28.1217) (524288,14.3352) (1048576,7.97673) (2097152,5.67928) (4194304,3.22785) (8388608,2.02894) (16777216,1.57046) (33554432,1.00511) (67108864,1.04) (1.34218e+08,1.19468) (2.68435e+08,1.17095) (5.36871e+08,1.10894) (1.07374e+09,1.10449) (2.14748e+09,1.08671) (4.29497e+09,2.1169) (8.58993e+09,2.11846) (1.71799e+10,1.95256) };
      \addlegendentry{\radixregion};
      \addplot coordinates { (8192,37.8489) (16384,36.6508) (32768,29.7578) (65536,31.9449) (131072,28.7984) (262144,406.903) (524288,141.64) (1048576,80.7396) (2097152,34.639) (4194304,19.2176) (8388608,6.02951) (16777216,2.37373) (33554432,4.30059) (67108864,2.89255) (1.34218e+08,2.242) (2.68435e+08,1.58004) (5.36871e+08,1.16829) (1.07374e+09,0.946814) (2.14748e+09,0.738893) (4.29497e+09,0.775105) (8.58993e+09,1.82392) (1.71799e+10,1.64434) };
      \addlegendentry{\compiparassrsort};

      \legend{}
      
      \coordinate (c2) at (rel axis cs:1,1);
      
      \nextgroupplot[
      y unit=ns,
      title=\distduplicatesroot \ulong]
      %% MULTIPLOT(algo|ptitle) select title as ptitle, size as x, 1000000.0 * milli * threads / (dsize * size * log(2, size)) as y, MULTIPLOT
      %% from pavgnames
      %% where algo not like 'gnuparallel' and algo not like 'tbbparallelsort' and  algo not like 'ps4oparallel' and machine like 'i10pc135' and datatype like 'uint64' and size >= 2^13
      %% and gen like 'rootdupls'
      %% and (algo not like 'tbbparallelsort' or 1000000.0 * milli * threads / (dsize * size * log(2, size)) <= 2)
      %% order by titleorder, x
      \addplot coordinates { (8192,59.1628) (16384,50.354) (32768,42.1661) (65536,39.4633) (131072,40.9771) (262144,156.617) (524288,134.8) (1048576,59.8525) (2097152,20.4553) (4194304,8.38434) (8388608,3.85275) (16777216,2.20566) (33554432,1.12066) (67108864,0.683765) (1.34218e+08,0.480851) (2.68435e+08,0.439137) (5.36871e+08,0.363777) (1.07374e+09,0.346391) (2.14748e+09,0.330698) (4.29497e+09,0.339541) (8.58993e+09,0.325587) (1.71799e+10,0.328449) };
      \addlegendentry{\compiparassssort};
      \addplot coordinates { (8192,74.1979) (16384,255.951) (32768,97.3478) (65536,36.1873) (131072,23.7192) (262144,7.48796) (524288,4.69639) (1048576,5.02382) (2097152,2.1755) (4194304,1.6321) (8388608,1.24079) (16777216,0.938009) (33554432,0.782048) (67108864,0.677873) (1.34218e+08,0.680535) (2.68435e+08,0.586441) (5.36871e+08,0.6705) (1.07374e+09,0.61972) (2.14748e+09,0.663012) (4.29497e+09,0.623167) (8.58993e+09,0.686421) (1.71799e+10,0.611639) };
      \addlegendentry{\compppbbs};
      \addplot coordinates { (8192,5321.55) (16384,2821.83) (32768,1185.29) (65536,665.457) (131072,373.075) (262144,238.03) (524288,1225.24) (1048576,625.477) (2097152,208.256) (4194304,135.613) (8388608,41.9328) (16777216,30.7078) (33554432,8.2314) (67108864,7.89251) (1.34218e+08,2.99216) (2.68435e+08,3.31856) (5.36871e+08,1.55576) (1.07374e+09,2.09192) (2.14748e+09,1.78924) (4.29497e+09,1.87618) (8.58993e+09,1.14255) (1.71799e+10,1.42091) };
      \addlegendentry{\comppbalancedsort};
      \addplot coordinates { (8192,130.22) (16384,184.939) (32768,142.332) (65536,84.3673) (131072,51.8247) (262144,42.8172) (524288,23.2574) (1048576,16.8276) (2097152,13.1919) (4194304,12.0247) (8388608,12.0475) (16777216,9.54215) (33554432,8.18604) (67108864,7.13615) (1.34218e+08,8.34051) (2.68435e+08,7.81841) (5.36871e+08,7.40017) (1.07374e+09,6.83777) (2.14748e+09,6.60677) (4.29497e+09,9.73538) (8.58993e+09,9.73853) };
      \addlegendentry{\radixppbbr};
      \addplot coordinates { (8192,246384) (16384,115758) (32768,53744.5) (65536,25151.8) (131072,13119.8) (262144,6205.97) (524288,3267.65) (1048576,1676.45) (2097152,869.974) (4194304,483.947) (8388608,290.891) (16777216,156.846) (33554432,97.1009) (67108864,62.0186) (1.34218e+08,39.6939) (2.68435e+08,6.33006) (5.36871e+08,3.20996) (1.07374e+09,4.57471) (2.14748e+09,2.60381) (4.29497e+09,4.44129) (8.58993e+09,1.9771) (1.71799e+10,3.58655) };
      \addlegendentry{\radixraduls};
      \addplot coordinates { (8192,242.483) (16384,121.719) (32768,1350.29) (65536,173.772) (131072,113.329) (262144,39.1426) (524288,24.2266) (1048576,9.9744) (2097152,6.13054) (4194304,3.13292) (8388608,2.00308) (16777216,1.31803) (33554432,1.2037) (67108864,1.33541) (1.34218e+08,1.2637) (2.68435e+08,1.30275) (5.36871e+08,1.33759) (1.07374e+09,1.30381) (2.14748e+09,1.31454) (4.29497e+09,1.7439) (8.58993e+09,2.00601) (1.71799e+10,2.23569) };
      \addlegendentry{\radixregion};
      \addplot coordinates { (8192,33.0386) (16384,31.3056) (32768,26.0437) (65536,28.7056) (131072,28.867) (262144,373.785) (524288,211.988) (1048576,83.619) (2097152,26.9457) (4194304,16.6225) (8388608,10.3705) (16777216,5.41091) (33554432,2.68463) (67108864,1.74392) (1.34218e+08,1.22591) (2.68435e+08,0.875987) (5.36871e+08,0.985591) (1.07374e+09,1.04252) (2.14748e+09,0.877907) (4.29497e+09,2.39513) (8.58993e+09,1.9084) (1.71799e+10,1.84729) };
      \addlegendentry{\compiparassrsort};

      \legend{}

      \nextgroupplot[
      title=\distduplicatestwice \ulong,
      ]
      %% MULTIPLOT(algo|ptitle) select title as ptitle, size as x, 1000000.0 * milli * threads / (dsize * size * log(2, size)) as y, MULTIPLOT
      %% from pavgnames
      %% where algo not like 'gnuparallel' and algo not like 'tbbparallelsort' and  algo not like 'ps4oparallel' and machine like 'i10pc135' and datatype like 'uint64' and size >= 2^13
      %% and (algo not like 'tbbparallelsort' or 1000000.0 * milli * threads / (dsize * size * log(2, size)) <= 2)
      %% and gen like 'twicedupes'
      %% order by titleorder, x
      \addplot coordinates { (8192,112.057) (16384,94.9986) (32768,89.406) (65536,69.7236) (131072,60.5495) (262144,217.608) (524288,73.4825) (1048576,42.9745) (2097152,23.9911) (4194304,10.2894) (8388608,4.1209) (16777216,2.37384) (33554432,1.40907) (67108864,1.07402) (1.34218e+08,0.759337) (2.68435e+08,0.618074) (5.36871e+08,0.545789) (1.07374e+09,0.484044) (2.14748e+09,0.471393) (4.29497e+09,0.468308) (8.58993e+09,0.469725) (1.71799e+10,0.468599) };
      \addlegendentry{\compiparassssort};
      \addplot coordinates { (8192,278.028) (16384,190.716) (32768,77.535) (65536,32.6565) (131072,17.6605) (262144,17.9793) (524288,6.88166) (1048576,6.26101) (2097152,4.57634) (4194304,2.5945) (8388608,2.06417) (16777216,1.61755) (33554432,1.4094) (67108864,1.29221) (1.34218e+08,1.28809) (2.68435e+08,1.03381) (5.36871e+08,1.11337) (1.07374e+09,1.04256) (2.14748e+09,1.13408) (4.29497e+09,1.06976) (8.58993e+09,1.14105) (1.71799e+10,1.0647) };
      \addlegendentry{\compppbbs};
      \addplot coordinates { (8192,5938.92) (16384,2624.62) (32768,1340.58) (65536,671.286) (131072,299.717) (262144,158.65) (524288,672.129) (1048576,279.589) (2097152,138.62) (4194304,66.3885) (8388608,32.5147) (16777216,15.9564) (33554432,7.82196) (67108864,4.38449) (1.34218e+08,3.07617) (2.68435e+08,2.16626) (5.36871e+08,1.81517) (1.07374e+09,1.95367) (2.14748e+09,1.74417) (4.29497e+09,1.76554) (8.58993e+09,1.68465) (1.71799e+10,1.69387) };
      \addlegendentry{\comppbalancedsort};
      \addplot coordinates { (8192,218.662) (16384,284.131) (32768,155.413) (65536,103.758) (131072,69.5928) (262144,55.2763) (524288,32.9715) (1048576,23.7869) (2097152,20.1461) (4194304,20.3925) (8388608,17.0807) (16777216,14.2341) (33554432,12.1453) (67108864,10.1347) (1.34218e+08,9.46662) (2.68435e+08,13.6362) (5.36871e+08,11.5892) (1.07374e+09,7.63939) (2.14748e+09,6.06144) (4.29497e+09,8.14466) (8.58993e+09,8.00359) };
      \addlegendentry{\radixppbbr};
      \addplot coordinates { (8192,219754) (16384,102007) (32768,47176.2) (65536,22305.8) (131072,12011) (262144,7067.89) (524288,4664.56) (1048576,3448.73) (2097152,2768.43) (4194304,614.288) (8388608,90.8211) (16777216,44.8087) (33554432,25.0539) (67108864,16.1467) (1.34218e+08,11.798) (2.68435e+08,9.57593) (5.36871e+08,8.06599) (1.07374e+09,3.23934) (2.14748e+09,1.90205) (4.29497e+09,1.85911) (8.58993e+09,1.53449) (1.71799e+10,1.59772) };
      \addlegendentry{\radixraduls};
      \addplot coordinates { (8192,299.706) (16384,154.877) (32768,1189.71) (65536,1131.16) (131072,155.244) (262144,75.0976) (524288,35.7208) (1048576,16.031) (2097152,12.3282) (4194304,6.45004) (8388608,4.87064) (16777216,4.13783) (33554432,1.57012) (67108864,1.39708) (1.34218e+08,1.34054) (2.68435e+08,1.41226) (5.36871e+08,1.31189) (1.07374e+09,1.15408) (2.14748e+09,1.16444) (4.29497e+09,1.20669) (8.58993e+09,1.2795) (1.71799e+10,1.17295) };
      \addlegendentry{\radixregion};
      \addplot coordinates { (8192,46.0622) (16384,39.9824) (32768,37.017) (65536,43.3303) (131072,40.3105) (262144,234.212) (524288,90.9794) (1048576,54.2473) (2097152,23.6873) (4194304,11.3829) (8388608,4.743) (16777216,2.8224) (33554432,3.85889) (67108864,2.7886) (1.34218e+08,1.77078) (2.68435e+08,1.39233) (5.36871e+08,1.31365) (1.07374e+09,1.29201) (2.14748e+09,0.832344) (4.29497e+09,0.817234) (8.58993e+09,2.04957) (1.71799e+10,1.81996) };
      \addlegendentry{\compiparassrsort};

      \legend{}
      
      \nextgroupplot[
      every axis y label/.append style={at=(ticklabel cs:1.1)},
      ylabel={Running time $t/ Dn \log_2 n$},
      title=\distexpo \ulong]
      %% MULTIPLOT(algo|ptitle) select title as ptitle, size as x, 1000000.0 * milli * threads / (dsize * size * log(2, size)) as y, MULTIPLOT
      %% from pavgnames
      %% where algo not like 'gnuparallel' and algo not like 'tbbparallelsort' and  algo not like 'ps4oparallel' and machine like 'i10pc135' and datatype like 'uint64' and size >= 2^13
      %% and (algo not like 'tbbparallelsort' or 1000000.0 * milli * threads / (dsize * size * log(2, size)) <= 2)
      %% and gen like 'exponential'
      %% order by titleorder, x
      \addplot coordinates { (8192,78.6239) (16384,69.6803) (32768,60.1106) (65536,52.1306) (131072,49.6372) (262144,263.26) (524288,124.722) (1048576,70.0695) (2097152,28.3434) (4194304,12.1705) (8388608,6.41586) (16777216,2.97896) (33554432,2.21435) (67108864,1.11083) (1.34218e+08,0.806227) (2.68435e+08,0.601186) (5.36871e+08,0.482095) (1.07374e+09,0.430518) (2.14748e+09,0.422388) (4.29497e+09,0.403156) (8.58993e+09,0.388266) (1.71799e+10,0.385485) };
      \addlegendentry{\compiparassssort};
      \addplot coordinates { (8192,421.732) (16384,220.362) (32768,104.421) (65536,35.963) (131072,23.3686) (262144,9.21415) (524288,5.74547) (1048576,4.08153) (2097152,3.29122) (4194304,2.45763) (8388608,1.82647) (16777216,1.42706) (33554432,1.2435) (67108864,1.09359) (1.34218e+08,1.08587) (2.68435e+08,0.86765) (5.36871e+08,0.956743) (1.07374e+09,0.877399) (2.14748e+09,0.959364) (4.29497e+09,0.904741) (8.58993e+09,0.939472) (1.71799e+10,0.900236) };
      \addlegendentry{\compppbbs};
      \addplot coordinates { (8192,5731.52) (16384,2689.26) (32768,1285.15) (65536,675.541) (131072,381.897) (262144,364.255) (524288,679.996) (1048576,347.515) (2097152,140.892) (4194304,76.2227) (8388608,29.5288) (16777216,15.3473) (33554432,8.18662) (67108864,5.05767) (1.34218e+08,3.22824) (2.68435e+08,2.58699) (5.36871e+08,2.09805) (1.07374e+09,2.31076) (2.14748e+09,1.96334) (4.29497e+09,1.74796) (8.58993e+09,1.95681) (1.71799e+10,1.64948) };
      \addlegendentry{\comppbalancedsort};
      \addplot coordinates { (8192,149.037) (16384,138.212) (32768,74.6947) (65536,49.9647) (131072,32.0194) (262144,13.9849) (524288,7.19141) (1048576,4.61235) (2097152,2.7673) (4194304,2.08076) (8388608,1.41822) (16777216,0.933552) (33554432,0.781738) (67108864,0.703963) (1.34218e+08,0.653702) (2.68435e+08,0.673552) (5.36871e+08,0.830364) (1.07374e+09,1.13776) (2.14748e+09,0.872866) (4.29497e+09,0.905807) (8.58993e+09,0.9937) (1.71799e+10,1.25383) };
      \addlegendentry{\radixppbbr};
      \addplot coordinates { (8192,85975.8) (16384,78288.1) (32768,60313.1) (65536,18712.5) (131072,8157.97) (262144,2996.24) (524288,1175.29) (1048576,494.54) (2097152,227.92) (4194304,111.45) (8388608,51.648) (16777216,25.0987) (33554432,12.0227) (67108864,6.23544) (1.34218e+08,3.79023) (2.68435e+08,2.69812) (5.36871e+08,1.98709) (1.07374e+09,1.71986) (2.14748e+09,1.91966) (4.29497e+09,1.79696) (8.58993e+09,1.81121) (1.71799e+10,1.63873) };
      \addlegendentry{\radixraduls};
      \addplot coordinates { (8192,728.617) (16384,644.14) (32768,1876.28) (65536,1090.72) (131072,589.771) (262144,308.704) (524288,146.931) (1048576,67.1927) (2097152,30.9832) (4194304,15.4567) (8388608,7.93507) (16777216,4.18209) (33554432,2.8858) (67108864,2.27722) (1.34218e+08,1.73306) (2.68435e+08,1.54367) (5.36871e+08,1.43159) (1.07374e+09,1.38178) (2.14748e+09,1.3391) (4.29497e+09,1.35363) (8.58993e+09,1.24966) (1.71799e+10,1.22996) };
      \addlegendentry{\radixregion};
      \addplot coordinates { (8192,66.7775) (16384,68.4433) (32768,66.4563) (65536,53.1917) (131072,47.0217) (262144,211.162) (524288,78.7029) (1048576,41.5774) (2097152,25.6397) (4194304,9.76731) (8388608,5.16912) (16777216,2.71231) (33554432,1.4231) (67108864,1.07886) (1.34218e+08,0.796914) (2.68435e+08,0.642408) (5.36871e+08,0.571543) (1.07374e+09,0.544025) (2.14748e+09,0.508667) (4.29497e+09,0.48904) (8.58993e+09,0.478669) (1.71799e+10,0.452092) };
      \addlegendentry{\compiparassrsort};

      \legend{}
      
      \nextgroupplot[
      title=\distzipf \ulong,
      legend to name=legendplotp135,
      legend style={at={($(0,0)+(1cm,1cm)$)},legend columns=5,fill=none,draw=black,anchor=center,align=center},
      ]
      %% MULTIPLOT(algo|ptitle) select title as ptitle, size as x, 1000000.0 * milli * threads / (dsize * size * log(2, size)) as y, MULTIPLOT
      %% from pavgnames
      %% where algo not like 'gnuparallel' and algo not like 'tbbparallelsort' and  algo not like 'ps4oparallel' and machine like 'i10pc135' and datatype like 'uint64' and size >= 2^13
      %% and (algo not like 'tbbparallelsort' or 1000000.0 * milli * threads / (dsize * size * log(2, size)) <= 2)
      %% and gen like 'zipf'
      %% order by titleorder, x
      \addplot coordinates { (8192,100.394) (16384,78.6747) (32768,79.5655) (65536,66.4104) (131072,60.5979) (262144,186.888) (524288,128.216) (1048576,31.7506) (2097152,16.3242) (4194304,7.51268) (8388608,5.00603) (16777216,2.53728) (33554432,1.37665) (67108864,0.83274) (1.34218e+08,0.603974) (2.68435e+08,0.516284) (5.36871e+08,0.426743) (1.07374e+09,0.38403) (2.14748e+09,0.377219) (4.29497e+09,0.34582) (8.58993e+09,0.331223) (1.71799e+10,0.342349) };
      \addlegendentry{\compiparassssort};
      \addplot coordinates { (8192,287.319) (16384,217.732) (32768,89.886) (65536,33.6987) (131072,17.4948) (262144,10.438) (524288,6.41684) (1048576,4.46276) (2097152,3.31628) (4194304,2.59523) (8388608,1.93929) (16777216,1.476) (33554432,1.26873) (67108864,1.09335) (1.34218e+08,1.09436) (2.68435e+08,0.84399) (5.36871e+08,0.906998) (1.07374e+09,0.827542) (2.14748e+09,0.897801) (4.29497e+09,0.847914) (8.58993e+09,0.751154) (1.71799e+10,0.732269) };
      \addlegendentry{\compppbbs};
      \addplot coordinates { (8192,5007.52) (16384,2584) (32768,1189.03) (65536,613.392) (131072,321.2) (262144,186.698) (524288,658.059) (1048576,298.007) (2097152,137.544) (4194304,63.2359) (8388608,31.6318) (16777216,14.6121) (33554432,8.52409) (67108864,4.69806) (1.34218e+08,3.44891) (2.68435e+08,2.70279) (5.36871e+08,2.27813) (1.07374e+09,2.22235) (2.14748e+09,1.84926) (4.29497e+09,1.6228) (8.58993e+09,2.03846) (1.71799e+10,1.74656) };
      \addlegendentry{\comppbalancedsort};
      \addplot coordinates { (8192,299.357) (16384,198.742) (32768,187.688) (65536,122.051) (131072,87.2171) (262144,67.9516) (524288,47.3448) (1048576,37.4863) (2097152,31.9923) (4194304,29.0353) (8388608,25.2899) (16777216,19.7788) (33554432,16.832) (67108864,14.8356) (1.34218e+08,14.2173) (2.68435e+08,14.7988) (5.36871e+08,14.0334) (1.07374e+09,12.5919) (2.14748e+09,12.2783) (4.29497e+09,14.3835) (8.58993e+09,13.6083) };
      \addlegendentry{\radixppbbr};
      \addplot coordinates { (8192,346068) (16384,252292) (32768,159947) (65536,75243.2) (131072,35330.4) (262144,16493.9) (524288,7836.8) (1048576,3711.35) (2097152,1719.97) (4194304,815.536) (8388608,388.955) (16777216,184.275) (33554432,86.4858) (67108864,40.3981) (1.34218e+08,20.4933) (2.68435e+08,10.8575) (5.36871e+08,5.90961) (1.07374e+09,3.84055) (2.14748e+09,2.73119) (4.29497e+09,2.2929) (8.58993e+09,1.99067) (1.71799e+10,1.88048) };
      \addlegendentry{\radixraduls};
      \addplot coordinates { (8192,621.886) (16384,492.815) (32768,574.053) (65536,656.894) (131072,391.17) (262144,214.335) (524288,107.658) (1048576,46.3924) (2097152,22.4482) (4194304,11.1854) (8388608,5.73064) (16777216,3.33193) (33554432,2.5507) (67108864,2.01541) (1.34218e+08,1.61326) (2.68435e+08,1.38927) (5.36871e+08,1.38145) (1.07374e+09,1.3703) (2.14748e+09,1.24758) (4.29497e+09,1.24642) (8.58993e+09,1.08205) (1.71799e+10,1.08667) };
      \addlegendentry{\radixregion};
      \addplot coordinates { (8192,82.0837) (16384,71.8805) (32768,56.0155) (65536,54.9364) (131072,49.1373) (262144,247.606) (524288,126.793) (1048576,60.4948) (2097152,28.7568) (4194304,12.7938) (8388608,6.27534) (16777216,4.00155) (33554432,3.26906) (67108864,2.6991) (1.34218e+08,2.07427) (2.68435e+08,1.94314) (5.36871e+08,1.89042) (1.07374e+09,1.89415) (2.14748e+09,1.80563) (4.29497e+09,1.89073) (8.58993e+09,1.7583) (1.71799e+10,1.72727) };
      \addlegendentry{\compiparassrsort};

      \nextgroupplot[
      xlabel={Item count $n$},
      title=\distuniform \pair,
      xmax=2^34.5,
      xmin=2^16,
      xmax=2^32,
      xtick={2^16, 2^20, 2^24, 2^28, 2^32},
      ]
      %% MULTIPLOT(algo|ptitle) select title as ptitle, size as x, 1000000.0 * milli * threads / (dsize * size * log(2, size)) as y, MULTIPLOT
      %% from pavgnames
      %% where algo not like 'gnuparallel' and algo not like 'tbbparallelsort' and  algo not like 'ps4oparallel' and machine like 'i10pc135' and datatype like 'pair' and size >= 2^13
      %% and gen like 'random'
      %% and (algo not like 'tbbparallelsort' or 1000000.0 * milli * threads / (dsize * size * log(2, size)) <= 2)
      %% and (algo not like 'raduls' or 1000000.0 * milli * threads / (dsize * size * log(2, size)) <= 2)
      %% order by titleorder, x
      \addplot coordinates { (8192,63.0541) (16384,50.9882) (32768,50.678) (65536,43.6702) (131072,191.132) (262144,72.2395) (524288,42.9851) (1048576,17.4628) (2097152,7.89117) (4194304,3.85844) (8388608,2.2815) (16777216,1.27097) (33554432,0.914543) (67108864,0.682905) (1.34218e+08,0.590768) (2.68435e+08,0.483892) (5.36871e+08,0.431905) (1.07374e+09,0.425087) (2.14748e+09,0.422744) (4.29497e+09,0.419995) (8.58993e+09,0.419236) };
      \addlegendentry{\compiparassssort};
      \addplot coordinates { (8192,132.352) (16384,116.169) (32768,38.2842) (65536,19.0563) (131072,16.4224) (262144,6.95444) (524288,3.86279) (1048576,3.71743) (2097152,2.25894) (4194304,1.73473) (8388608,1.30495) (16777216,1.1543) (33554432,1.03045) (67108864,1.37683) (1.34218e+08,1.1686) (2.68435e+08,1.4917) (5.36871e+08,1.21787) (1.07374e+09,1.44933) (2.14748e+09,1.21083) (4.29497e+09,1.43691) (8.58993e+09,1.19537) };
      \addlegendentry{\compppbbs};
      \addplot coordinates { (8192,2837.4) (16384,1313.19) (32768,611.435) (65536,305.015) (131072,149.712) (262144,105.468) (524288,318.101) (1048576,149.682) (2097152,74.1336) (4194304,32.6404) (8388608,16.0494) (16777216,7.15702) (33554432,3.9981) (67108864,2.15002) (1.34218e+08,1.47394) (2.68435e+08,1.18148) (5.36871e+08,1.78326) (1.07374e+09,1.50471) (2.14748e+09,1.43969) (4.29497e+09,1.51469) (8.58993e+09,1.45354) };
      \addlegendentry{\comppbalancedsort};
      \addplot coordinates { (8192,137.674) (16384,113.374) (32768,35.9205) (65536,15.0001) (131072,8.03171) (262144,5.58596) (524288,3.76551) (1048576,2.9279) (2097152,1.7648) (4194304,1.49132) (8388608,1.04071) (16777216,0.687356) (33554432,0.602978) (67108864,0.581076) (1.34218e+08,0.54143) (2.68435e+08,0.527188) (5.36871e+08,0.826777) (1.07374e+09,0.509188) (2.14748e+09,0.565277) (4.29497e+09,0.588675) (8.58993e+09,0.693666) };
      \addlegendentry{\radixppbbr};
      \addplot coordinates { (8192,255.919) (16384,245.72) (32768,1006.8) (65536,570.371) (131072,306.941) (262144,148.877) (524288,67.4927) (1048576,31.6112) (2097152,15.8745) (4194304,8.21273) (8388608,4.30867) (16777216,3.20046) (33554432,2.32169) (67108864,1.88372) (1.34218e+08,1.58408) (2.68435e+08,1.44332) (5.36871e+08,1.34254) (1.07374e+09,1.27774) (2.14748e+09,1.23528) (4.29497e+09,1.09331) (8.58993e+09,1.2414) };
      \addlegendentry{\radixregion};
      \addplot coordinates { (8192,38.9416) (16384,35.962) (32768,42.292) (65536,34.9535) (131072,171.846) (262144,80.2205) (524288,46.4956) (1048576,20.0035) (2097152,13.0411) (4194304,5.28834) (8388608,2.33025) (16777216,1.37948) (33554432,0.818762) (67108864,0.604897) (1.34218e+08,0.491606) (2.68435e+08,0.439197) (5.36871e+08,0.422005) (1.07374e+09,0.401906) (2.14748e+09,0.419293) (4.29497e+09,0.394214) (8.58993e+09,0.389512) };
      \addlegendentry{\compiparassrsort};

      \legend{}
      
      \nextgroupplot[
      xlabel={Item count $n$},
      title=\distuniform \bytes,
      every legend/.append style={at=(ticklabel cs:1.1)},
      xmax=2^31.5,
      xmin=2^14,
      xmax=2^30,
      xtick={2^14, 2^18, 2^22, 2^26, 2^30},
      ]
      %% MULTIPLOT(algo|ptitle) select title as ptitle, size as x, 1000000.0 * milli * threads / (dsize * size * log(2, size)) as y, MULTIPLOT
      %% from pavgnames
      %% where algo not like 'gnuparallel' and algo not like 'tbbparallelsort' and  algo not like 'ps4oparallel' and machine like 'i10pc135' and datatype like 'byte' and size >= 2^13
      %% and gen like 'byte'
      %% and (algo not like 'tbbparallelsort' or 1000000.0 * milli * threads / (dsize * size * log(2, size)) <= 2)
      %% and (algo not like 'gnuquicksortbalanced' or 1000000.0 * milli * threads / (dsize * size * log(2, size)) <= 2)
      %% and (algo not like 'gnuquicksortunbalanced' or 1000000.0 * milli * threads / (dsize * size * log(2, size)) <= 2)
      %% order by titleorder, x
      \addplot coordinates { (8192,30.7207) (16384,395.817) (32768,212.425) (65536,68.5549) (131072,26.9105) (262144,14.9205) (524288,6.11554) (1048576,4.31258) (2097152,1.51241) (4194304,0.959983) (8388608,0.81774) (16777216,0.698877) (33554432,0.509012) (67108864,0.453324) (1.34218e+08,0.441197) (2.68435e+08,0.424462) (5.36871e+08,0.407417) (1.07374e+09,0.414346) };
      \addlegendentry{\compiparassssort};
      \addplot coordinates { (8192,29.7074) (16384,21.5435) (32768,6.46749) (65536,4.32617) (131072,3.02061) (262144,1.69854) (524288,1.24283) (1048576,1.08811) (2097152,0.838657) (4194304,0.63167) (8388608,0.551358) (16777216,0.515986) (33554432,0.487298) (67108864,0.535411) (1.34218e+08,0.529589) (2.68435e+08,0.583544) (5.36871e+08,0.596929) (1.07374e+09,0.612939) };
      \addlegendentry{\compppbbs};

      \legend{}
      
    \end{groupplot}
    \coordinate (c3) at ($(c1)!.5!(c2)$);
    \node[below] at (c3 |- current bounding box.south)
    {\pgfplotslegendfromname{legendplotp135}};
  \end{tikzpicture}

  \caption{
    Running times of parallel algorithms on different input distributions and data types of size $D$ executed on machine \pcintellargefour.
    A horizontal (vertical) line in the top right corner indicates that the algorithm's running time was too large for the plot (that the algorithm's interface does not accept the data type).
  }
  \label{fig:par rt distr types 135}
\end{figure}

\end{document}

%%%%%%%%%%%%%%%%%%%%%%%%%%%%%%%%%%%%%%%%%%%%%%%%%%%%%%%%%%%%%%%%%%%%%%%%%%%%%%%%
